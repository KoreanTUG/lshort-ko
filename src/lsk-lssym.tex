%%%%%%%%%%%%%%%%%%%%%%%%%%%%%%%%%%%%%%%%%%%%%%%%%%%%%%%%%%%%%%%%%
% Contents: TeX and LaTeX and AMS symbols for Maths
% $Id$
%%%%%%%%%%%%%%%%%%%%%%%%%%%%%%%%%%%%%%%%%%%%%%%%%%%%%%%%%%%%%%%%%


%\section{List of Mathematical Symbols}  \label{symbols}
\section{수학 기호 목록} \label{symbols}

%The following tables demonstrate all the symbols normally accessible
%from \emph{math mode}.
다음 수학기호 목록은 \emph{수식모드(math mode)}에서 사용가능한 기호문자를 
열거한 것이다.

%
% Conditional Text in case the AMS Fonts are installed
%
% Note that some tables show symbols only accessible after loading the \pai{amssymb}
% package in the preamble of your document\footnote{The tables were derived
%   from \texttt{symbols.tex} by David~Carlisle and subsequently changed
% extensively as suggested by Josef~Tkadlec.}. If the \AmS{} package and
% fonts are not installed on your system, have a look at
% \CTANref|CTAN:pkg/amslatex|. An even more comprehensive list of
% symbols can be found at \CTANref|CTAN:info/symbols/comprehensive|.
아래 기호 중의 일부는 \pai{amssymb} 패키지를 로드하여야 사용할 수 있다.\footnote{%
        이 표는 David Carlisle이 처음 작성하고 그 후 Josef~Ikadlec이 
        확장하여 수정한 \texttt{symbols.tex}에서 가져온 것이다.
}
\AmS{} 패키지와 그 폰트에 관해서는 \CTAN|pkg/amslatex|을 보라. 
완전한 기호문자 일람표를 보려면 \CTAN|info/symbols/comprehensive|를 참조하라.

\vfill

\begin{table}[!h]
%\caption{Math Mode Accents.}  \label{mathacc}
\caption{수식모드의 액센트} \label{mathacc}
\begin{lsksymbols}{*3{cl}}
\mstW{\hat}{a}   & \mstW{\check}{a} & \mstW{\tilde}{a}       \\
\mstW{\grave}{a} & \mstW{\dot}{a}   & \mstW{\ddot}{a}        \\
\mstW{\bar}{a}   & \mstW{\vec}{a}   & \mstW{\widehat}{AAA}   \\
\mstW{\acute}{a} & \mstW{\breve}{a} & \mstW{\widetilde}{AAA} \\
\mstW{\mathring}{a}
\end{lsksymbols}
\end{table}


\begin{table}[!h]
%\caption{Greek Letters.} \label{greekletters}
\caption{그리스 문자} \label{greekletters}
\bigskip
% There is no uppercase of some of the letters like \lskci{Alpha}, \lskci{Beta} and so
% on, because they look the same as normal roman letters: A, B\ldots
대문자 Alpha를 위한 \cs{Alpha}나 \cs{Beta}라는 명령은 없다. 왜냐하면 이 글자들은
로마자 공통이고 \texttt{A, B}와 같이 입력하여 A, B를 얻을 수 있기 때문이다.
\begin{lsksymbols}{*4{cl}}
 \mstX{\alpha}     & \mstX{\theta}     & \mstX{o}          & \mstX{\upsilon}  \\
 \mstX{\beta}      & \mstX{\vartheta}  & \mstX{\pi}        & \mstX{\phi}      \\
 \mstX{\gamma}     & \mstX{\iota}      & \mstX{\varpi}     & \mstX{\varphi}   \\
 \mstX{\delta}     & \mstX{\kappa}     & \mstX{\rho}       & \mstX{\chi}      \\
 \mstX{\epsilon}   & \mstX{\lambda}    & \mstX{\varrho}    & \mstX{\psi}      \\
 \mstX{\varepsilon}& \mstX{\mu}        & \mstX{\sigma}     & \mstX{\omega}    \\
 \mstX{\zeta}      & \mstX{\nu}        & \mstX{\varsigma}  &               \\
 \mstX{\eta}       & \mstX{\xi}        & \mstX{\tau} & \\
 \mstX{\Gamma}     & \mstX{\Lambda}    & \mstX{\Sigma}     & \mstX{\Psi}      \\
 \mstX{\Delta}     & \mstX{\Xi}        & \mstX{\Upsilon}   & \mstX{\Omega}    \\
 \mstX{\Theta}     & \mstX{\Pi}        & \mstX{\Phi}
\end{lsksymbols}
\end{table}



\begin{table}[!tbp]
% \caption{Binary Relations.} 
\caption{이항 관계 연산자}
\label{binaryrel}
\bigskip
% You can negate the following symbols by prefixing them with a \lskci{not} command.
다음 연산자 부호 앞에 \ci{not}를 붙이면 부정 기호를 얻을 수 있다.
\begin{lsksymbols}{*3{cl}}
 \mstX{<}           & \mstX{>}           & \mstX{=}          \\
 \mstX{\leq}or \verb|\le|   & \mstX{\geq}or \verb|\ge|   & \mstX{\equiv}     \\
 \mstX{\ll}         & \mstX{\gg}         & \mstX{\doteq}     \\
 \mstX{\prec}       & \mstX{\succ}       & \mstX{\sim}       \\
 \mstX{\preceq}     & \mstX{\succeq}     & \mstX{\simeq}     \\
 \mstX{\subset}     & \mstX{\supset}     & \mstX{\approx}    \\
 \mstX{\subseteq}   & \mstX{\supseteq}   & \mstX{\cong}      \\
 \mstX{\sqsubset}$^a$ & \mstX{\sqsupset}$^a$ & \mstX{\Join}$^a$    \\
 \mstX{\sqsubseteq} & \mstX{\sqsupseteq} & \mstX{\bowtie}    \\
 \mstX{\in}         & \mstX{\ni}, \verb|\owns|  & \mstX{\propto}    \\
 \mstX{\vdash}      & \mstX{\dashv}      & \mstX{\models}    \\
 \mstX{\mid}        & \mstX{\parallel}   & \mstX{\perp}      \\
 \mstX{\smile}      & \mstX{\frown}      & \mstX{\asymp}     \\
 \mstX{:}           & \mstX{\notin}      & \mstX{\neq}or \verb|\ne|
\end{lsksymbols}
%\centerline{\footnotesize $^a$Use the \textsf{latexsym} package to access this symbol}
\centerline{\footnotesize $^a$%
이 기호는 \textsf{latexsym} 패키지를 요구한다.
}
\end{table}

\begin{table}[!tbp]
%\caption{Binary Operators.}
\caption{이항 연산자}
\begin{lsksymbols}{*3{cl}}
 \mstX{+}              & \mstX{-}              & &                 \\
 \mstX{\pm}            & \mstX{\mp}            & \mstX{\triangleleft} \\
 \mstX{\cdot}          & \mstX{\div}           & \mstX{\triangleright}\\
 \mstX{\times}         & \mstX{\setminus}      & \mstX{\star}         \\
 \mstX{\cup}           & \mstX{\cap}           & \mstX{\ast}          \\
 \mstX{\sqcup}         & \mstX{\sqcap}         & \mstX{\circ}         \\
 \mstX{\vee}, \verb|\lor|     & \mstX{\wedge}, \verb|\land|  & \mstX{\bullet}       \\
 \mstX{\oplus}         & \mstX{\ominus}        & \mstX{\diamond}      \\
 \mstX{\odot}          & \mstX{\oslash}        & \mstX{\uplus}        \\
 \mstX{\otimes}        & \mstX{\bigcirc}       & \mstX{\amalg}        \\
 \mstX{\bigtriangleup} &\mstX{\bigtriangledown}& \mstX{\dagger}       \\
 \mstX{\lhd}$^a$         & \mstX{\rhd}$^a$         & \mstX{\ddagger}      \\
 \mstX{\unlhd}$^a$       & \mstX{\unrhd}$^a$       & \mstX{\wr}
\end{lsksymbols}

\end{table}

\begin{table}[!tbp]
% \caption{BIG Operators.}
\caption[큰 연산자]{\dotemph{큰} 연산자}
\begin{lsksymbols}{*4{cl}}
 \mstX{\sum}      & \mstX{\bigcup}   & \mstX{\bigvee}  \\
 \mstX{\prod}     & \mstX{\bigcap}   & \mstX{\bigwedge} \\
 \mstX{\coprod}   & \mstX{\bigsqcup} & \mstX{\biguplus} \\
 \mstX{\int}      & \mstX{\oint}     & \mstX{\bigodot} \\
 \mstX{\bigoplus} & \mstX{\bigotimes} & \\
\end{lsksymbols}

\end{table}


\begin{table}[!tbp]
% \caption{Arrows.} 
\caption{화살표}
\label{tab:arrows}
\begin{lsksymbols}{*2{cl}}
 \mstX{\leftarrow}or \verb|\gets|& \mstX{\longleftarrow} \\
 \mstX{\rightarrow}or \verb|\to|& \mstX{\longrightarrow} \\
 \mstX{\leftrightarrow}    & \mstX{\longleftrightarrow} \\
 \mstX{\Leftarrow}         & \mstX{\Longleftarrow}     \\
 \mstX{\Rightarrow}        & \mstX{\Longrightarrow}    \\
 \mstX{\Leftrightarrow}    & \mstX{\Longleftrightarrow}\\
 \mstX{\mapsto}            & \mstX{\longmapsto}        \\
 \mstX{\hookleftarrow}     & \mstX{\hookrightarrow}    \\
 \mstX{\leftharpoonup}     & \mstX{\rightharpoonup}    \\
 \mstX{\leftharpoondown}   & \mstX{\rightharpoondown}  \\
 \mstX{\rightleftharpoons} & \mstX{\iff}(bigger spaces) \\
 \mstX{\uparrow}   & \mstX{\downarrow} \\
 \mstX{\updownarrow} & \mstX{\Uparrow} \\
 \mstX{\Downarrow} &  \mstX{\Updownarrow} \\
 \mstX{\nearrow} &  \mstX{\searrow} \\
  \mstX{\swarrow} & \mstX{\nwarrow} \\
 \mstX{\leadsto}$^a$
\end{lsksymbols}
%\centerline{\footnotesize $^a$Use the \textsf{latexsym} package to access this symbol}
\centerline{\footnotesize $^a$해당 기호는 \textsf{latexsym} 패키지가 필요함}
\end{table}

\begin{table}[!tbp]
%\caption{Arrows as Accents.}  
\caption{문자의 위아래로 오는 화살표}
\label{arrowacc}
\begin{lsksymbols}{*2{cl}}
\mstW{\overrightarrow}{AB}     & \mstW{\underrightarrow}{AB}     \\
\mstW{\overleftarrow}{AB}      & \mstW{\underleftarrow}{AB}      \\
\mstW{\overleftrightarrow}{AB} & \mstW{\underleftrightarrow}{AB} \\
\end{lsksymbols}
\end{table}

\begin{table}[!tbp]
%\caption{Delimiters.}
\caption{여닫는 부호}
\label{tab:delimiters}
\begin{lsksymbols}{*3{cl}}
 \mstX{(}            & \mstX{)}            & \mstX{\uparrow} \\
 \mstX{[}or \verb|\lbrack|   & \mstX{]}or \verb|\rbrack|  & \mstX{\downarrow}   \\
 \mstX{\{}or \verb|\lbrace|  & \mstX{\}}or \verb|\rbrace|  & \mstX{\updownarrow} \\
 \mstX{\langle}      & \mstX{\rangle}      &  \mstX{\Uparrow} \\
 \mstX{|}or \verb|\vert| & \mstX{\|}or \verb|\Vert| & \mstX{\Downarrow} \\
  \mstX{/}            & \mstX{\backslash}   &   \mstX{\Updownarrow}  \\
 \mstX{\lfloor}      & \mstX{\rfloor}      &  \\
 \mstX{\rceil}       &  \mstX{\lceil}  &&\\
\end{lsksymbols}
\end{table}

\begin{table}[!tbp]
%\caption{Large Delimiters.}
\caption{큰 여닫는 부호}
\begin{lsksymbols}{*3{cl}}
 \mstY{\lgroup}      & \mstY{\rgroup}      & \mstY{\lmoustache}  \\
 \mstY{\arrowvert}   & \mstY{\Arrowvert}   & \mstY{\bracevert} \\
 \mstY{\rmoustache} \\
\end{lsksymbols}
\end{table}


\begin{table}[!tbp]
%\caption{Miscellaneous Symbols.}
\caption{기타 부호}
\begin{lsksymbols}{*4{cl}}
 \mstX{\dots}       & \mstX{\cdots}      & \mstX{\vdots}      & \mstX{\ddots}     \\
 \mstX{\hbar}       & \mstX{\imath}      & \mstX{\jmath}      & \mstX{\ell}       \\
 \mstX{\Re}         & \mstX{\Im}         & \mstX{\aleph}      & \mstX{\wp}        \\
 \mstX{\forall}     & \mstX{\exists}     & \mstX{\mho}$^a$      & \mstX{\partial}   \\
 \mstX{'}           & \mstX{\prime}      & \mstX{\emptyset}   & \mstX{\infty}     \\
 \mstX{\nabla}      & \mstX{\triangle}   & \mstX{\Box}$^a$     & \mstX{\Diamond}$^a$ \\
 \mstX{\bot}        & \mstX{\top}        & \mstX{\angle}      & \mstX{\surd}      \\
\mstX{\diamondsuit} & \mstX{\heartsuit}  & \mstX{\clubsuit}   & \mstX{\spadesuit} \\
 \mstX{\neg}or \verb|\lnot| & \mstX{\flat}       & \mstX{\natural}    & \mstX{\sharp}
\end{lsksymbols}
% \centerline{\footnotesize $^a$Use the \textsf{latexsym} package to access this symbol}
\centerline{\footnotesize $^a$해당 기호는 \textsf{latexsym} 패키지가 필요함}
\end{table}


\begin{table}[!tbp]
%\caption{Non-Mathematical Symbols.}
\caption{수학 기호가 아닌 것}
\bigskip
%These symbols can also be used in text mode.
다음 기호들은 텍스트 모드에서도 사용할 수 있다.
\begin{lsksymbols}{*4{cl}}
 \mstSC{\dag}  &  \mstSC{\S}  &  \mstSC{\copyright} &  \mstSC{\textregistered}  \\
 \mstSC{\ddag} &  \mstSC{\P}  &  \mstSC{\pounds}    &  \mstSC{\%}               \\
\end{lsksymbols}
\end{table}

%\clearpage

%
%
% If the AMS Stuff is not available, we drop out right here :-)
%

\begin{table}[!tbp]
%\caption{\AmS{} Delimiters.}
\caption{\AmS{}: 여닫는 부호}
\label{AMSD}
\bigskip
\begin{lsksymbols}{*4{cl}}
\mstX{\ulcorner}&\mstX{\urcorner}&\mstX{\llcorner}&\mstX{\lrcorner}\\
\mstX{\lvert}&\mstX{\rvert}&\mstX{\lVert}&\mstX{\rVert}
\end{lsksymbols}
\end{table}

\begin{table}[!tbp]
%\caption{\AmS{} Greek and Hebrew.}
\caption{\AmS{}: 그리스와 히브리 문자}
\begin{lsksymbols}{*5{cl}}
\mstX{\digamma}     &\mstX{\varkappa} & \mstX{\beth} &\mstX{\gimel} & \mstX{\daleth}
\end{lsksymbols}
\end{table}

\begin{table}[tbp]
%  \caption{Math Alphabets.} 
\caption{수학 알파벳}
\label{mathalpha}
% \bigskip See Table~\ref{mathfonts} on \pageref{mathfonts} for other math fonts.
\bigskip 그밖의 수학 폰트에 관해서는 \pageref{mathfonts}페이지의 표~\ref{mathfonts}\를 보라.
\begin{lsksymbols}{@{}*3l@{}}
%Example& Command &Required package\\
예문&명령&필요패키지 \\
\hline
\rule{0pt}{1.05em}$\mathrm{ABCDE abcde 1234}$
        & \verb|\mathrm{ABCDE abcde 1234}|
        &       \\
$\mathit{ABCDE abcde 1234}$
        & \verb|\mathit{ABCDE abcde 1234}|
        &       \\
$\mathnormal{ABCDE abcde 1234}$
        & \verb|\mathnormal{ABCDE abcde 1234}|
        &  \\
$\mathcal{ABCDE abcde 1234}$
        & \verb|\mathcal{ABCDE abcde 1234}|
        &  \\
$\mathscr{ABCDE abcde 1234}$
        &\verb|\mathscr{ABCDE abcde 1234}|
        &\pai{mathrsfs}\\
$\mathfrak{ABCDE abcde 1234}$
        & \verb|\mathfrak{ABCDE abcde 1234}|
        &\pai{amsfonts}  or \textsf{amssymb}  \\
$\mathbb{ABCDE abcde 1234}$
        & \verb|\mathbb{ABCDE abcde 1234}|
        &\pai{amsfonts}  or \textsf{amssymb} \\
\end{lsksymbols}
\end{table}

\begin{table}[!tbp]
%\caption{\AmS{} Binary Operators.}
\caption{\AmS{}: 이항 관계 연산자}
\begin{lsksymbols}{*3{cl}}
 \mstX{\dotplus}        & \mstX{\centerdot}      &       \\
 \mstX{\ltimes}         & \mstX{\rtimes}         & \mstX{\divideontimes} \\
 \mstX{\doublecup}      & \mstX{\doublecap}	   & \mstX{\smallsetminus} \\
 \mstX{\veebar}         & \mstX{\barwedge}       & \mstX{\doublebarwedge}\\
 \mstX{\boxplus}        & \mstX{\boxminus}       & \mstX{\circleddash}   \\
 \mstX{\boxtimes}       & \mstX{\boxdot}         & \mstX{\circledcirc}   \\
 \mstX{\intercal}       & \mstX{\circledast}     & \mstX{\rightthreetimes} \\
 \mstX{\curlyvee}       & \mstX{\curlywedge}     & \mstX{\leftthreetimes}
\end{lsksymbols}
\end{table}

\begin{table}[!tbp]
%\caption{\AmS{} Binary Relations.}
\caption{\AmS: 이항 연산자}
\begin{lsksymbols}{*3{cl}}
 \mstX{\lessdot}           & \mstX{\gtrdot}            & \mstX{\doteqdot} \\
 \mstX{\leqslant}          & \mstX{\geqslant}          & \mstX{\risingdotseq}     \\
 \mstX{\eqslantless}       & \mstX{\eqslantgtr}        & \mstX{\fallingdotseq}    \\
 \mstX{\leqq}              & \mstX{\geqq}              & \mstX{\eqcirc}           \\
 \mstX{\lll}or \verb|\llless| & \mstX{\ggg}            & \mstX{\circeq}  \\
 \mstX{\lesssim}           & \mstX{\gtrsim}            & \mstX{\triangleq}        \\
 \mstX{\lessapprox}        & \mstX{\gtrapprox}         & \mstX{\bumpeq}           \\
 \mstX{\lessgtr}           & \mstX{\gtrless}           & \mstX{\Bumpeq}           \\
 \mstX{\lesseqgtr}         & \mstX{\gtreqless}         & \mstX{\thicksim}         \\
 \mstX{\lesseqqgtr}        & \mstX{\gtreqqless}        & \mstX{\thickapprox}      \\
 \mstX{\preccurlyeq}       & \mstX{\succcurlyeq}       & \mstX{\approxeq}         \\
 \mstX{\curlyeqprec}       & \mstX{\curlyeqsucc}       & \mstX{\backsim}          \\
 \mstX{\precsim}           & \mstX{\succsim}           & \mstX{\backsimeq}        \\
 \mstX{\precapprox}        & \mstX{\succapprox}        & \mstX{\vDash}            \\
 \mstX{\subseteqq}         & \mstX{\supseteqq}         & \mstX{\Vdash}            \\
 \mstX{\shortparallel}     & \mstX{\Supset}            & \mstX{\Vvdash}           \\
 \mstX{\blacktriangleleft} & \mstX{\sqsupset}          & \mstX{\backepsilon}      \\
 \mstX{\vartriangleright}  & \mstX{\because}           & \mstX{\varpropto}        \\
 \mstX{\blacktriangleright}& \mstX{\Subset}            & \mstX{\between}          \\
 \mstX{\trianglerighteq}   & \mstX{\smallfrown}        & \mstX{\pitchfork}        \\
 \mstX{\vartriangleleft}   & \mstX{\shortmid} 	 & \mstX{\smallsmile} 	\\
 \mstX{\trianglelefteq}    & \mstX{\therefore} 	 & \mstX{\sqsubset}
\end{lsksymbols}
\end{table}

\begin{table}[!tbp]
%\caption{\AmS{} Arrows.}
\caption{\AmS: 화살표}
\begin{lsksymbols}{*2{cl}}
 \mstX{\dashleftarrow}      & \mstX{\dashrightarrow}     \\
 \mstX{\leftleftarrows}     & \mstX{\rightrightarrows}   \\
 \mstX{\leftrightarrows}    & \mstX{\rightleftarrows}    \\
 \mstX{\Lleftarrow}         & \mstX{\Rrightarrow}        \\
 \mstX{\twoheadleftarrow}   & \mstX{\twoheadrightarrow}  \\
 \mstX{\leftarrowtail}      & \mstX{\rightarrowtail}     \\
 \mstX{\leftrightharpoons}  & \mstX{\rightleftharpoons}  \\
 \mstX{\Lsh}                & \mstX{\Rsh}                \\
 \mstX{\looparrowleft}      & \mstX{\looparrowright}     \\
 \mstX{\curvearrowleft}     & \mstX{\curvearrowright}    \\
 \mstX{\circlearrowleft}    & \mstX{\circlearrowright}   \\
 \mstX{\multimap}  &  \mstX{\upuparrows}  \\
 \mstX{\downdownarrows} & \mstX{\upharpoonleft} \\
 \mstX{\upharpoonright} & \mstX{\downharpoonright} \\
 \mstX{\rightsquigarrow} & \mstX{\leftrightsquigarrow} \\
\end{lsksymbols}
\end{table}

\begin{table}[!tbp]
%\caption{\AmS{} Negated Binary Relations and Arrows.}
\caption{\AmS: 이항 연산자와 화살표의 부정}
\label{AMSNBR}
\begin{lsksymbols}{*3{cl}}
 \mstX{\nless}           & \mstX{\ngtr}            & \mstX{\varsubsetneqq}  \\
 \mstX{\lneq}            & \mstX{\gneq}            & \mstX{\varsupsetneqq}  \\
 \mstX{\nleq}            & \mstX{\ngeq}            & \mstX{\nsubseteqq}     \\
 \mstX{\nleqslant}       & \mstX{\ngeqslant}       & \mstX{\nsupseteqq}     \\
 \mstX{\lneqq}           & \mstX{\gneqq}           & \mstX{\nmid}           \\
 \mstX{\lvertneqq}       & \mstX{\gvertneqq}       & \mstX{\nparallel}      \\
 \mstX{\nleqq}           & \mstX{\ngeqq}           & \mstX{\nshortmid}      \\
 \mstX{\lnsim}           & \mstX{\gnsim}           & \mstX{\nshortparallel} \\
 \mstX{\lnapprox}        & \mstX{\gnapprox}        & \mstX{\nsim}           \\
 \mstX{\nprec}           & \mstX{\nsucc}           & \mstX{\ncong}          \\
 \mstX{\npreceq}         & \mstX{\nsucceq}         & \mstX{\nvdash}         \\
 \mstX{\precneqq}        & \mstX{\succneqq}        & \mstX{\nvDash}         \\
 \mstX{\precnsim}        & \mstX{\succnsim}        & \mstX{\nVdash}         \\
 \mstX{\precnapprox}     & \mstX{\succnapprox}     & \mstX{\nVDash}         \\
 \mstX{\subsetneq}       & \mstX{\supsetneq}       & \mstX{\ntriangleleft}  \\
 \mstX{\varsubsetneq}    & \mstX{\varsupsetneq}    & \mstX{\ntriangleright} \\
 \mstX{\nsubseteq}       & \mstX{\nsupseteq}       & \mstX{\ntrianglelefteq}\\
 \mstX{\subsetneqq}      & \mstX{\supsetneqq}      &\mstX{\ntrianglerighteq}\\[0.5ex]
 \mstX{\nleftarrow}      & \mstX{\nrightarrow}     & \mstX{\nleftrightarrow}\\
 \mstX{\nLeftarrow}      & \mstX{\nRightarrow}     & \mstX{\nLeftrightarrow}

\end{lsksymbols}
\end{table}

\begin{table}[!htp] \label{AMSmisc}
%\caption{\AmS{} Miscellaneous.}
\caption{\AmS: 기타}
\begin{lsksymbols}{*3{cl}}
 \mstX{\hbar}             & \mstX{\hslash}           & \mstX{\Bbbk}            \\
 \mstX{\square}           & \mstX{\blacksquare}      & \mstX{\circledS}        \\
 \mstX{\vartriangle}      & \mstX{\blacktriangle}    & \mstX{\complement}      \\
 \mstX{\triangledown}     &\mstX{\blacktriangledown} & \mstX{\Game}            \\
 \mstX{\lozenge}          & \mstX{\blacklozenge}     & \mstX{\bigstar}         \\
 \mstX{\angle}            & \mstX{\measuredangle}    & \\
 \mstX{\diagup}           & \mstX{\diagdown}         & \mstX{\backprime}       \\
 \mstX{\nexists}          & \mstX{\Finv}             & \mstX{\varnothing}      \\
 \mstX{\eth}              & \mstX{\sphericalangle}   & \mstX{\mho}
\end{lsksymbols}

\vspace*{10cm}
\end{table}

\clearpage

\endinput

%

% Local Variables:
% TeX-master: "lshort2e"
% mode: latex
% mode: flyspell
% End:
