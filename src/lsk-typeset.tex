%%%%%%%%%%%%%%%%%%%%%%%%%%%%%%%%%%%%%%%%%%%%%%%%%%%%%%%%%%%%%%%%%
% Contents: Typesetting Part of LaTeX2e Introduction
% $Id$
%%%%%%%%%%%%%%%%%%%%%%%%%%%%%%%%%%%%%%%%%%%%%%%%%%%%%%%%%%%%%%%%%
% \chapter{Typesetting Text}
\chapter{텍스트의 조판}

% \begin{intro}
%   After reading the previous chapter, you should know about the basic
%   stuff of which a \LaTeXe{} document is made. In this chapter I
%   will fill in the remaining structure you will need to know in order
%   to produce real world material.
% \end{intro}

\begin{intro}
  앞 장을 읽고서 \LaTeX{} 문서의 기초 사항을 알게 되었다. 이 장에서는 
  실제 제대로 된 문서를 작성하기 위해 알아두어야 할 문서 구조에 대하여 다룬다.
\end{intro}

% \section{The Structure of Text and Language}
% \secby{Hanspeter Schmid}{hanspi@schmid-werren.ch}
% The main point of writing a text, is to convey ideas, information, or
% knowledge to the reader.  The reader will understand the text better
% if these ideas are well-structured, and will see and feel this
% structure much better if the typographical form reflects the logical
% and semantic structure of the content.
\section{텍스트와 언어의 구조}
\secby{Hapspeter Schmid}{hanspi@schmid-werren.ch}%
글쓰기의 주목적은 저자의 사상, 정보, 지식을 전달하는 것이다. 내용이 잘 구조화되어 있을수록 
독자가 이해하기 쉽다. 또한 타이포그래피적 요소가 내용의 논리적 의미적 구조를 잘 반영하고 있을수록 
그 구조를 바로 파악하기 쉬워진다.

% \LaTeX{} is different from other typesetting systems in that you just
% have to tell it the logical and semantic structure of a text.  It
% then derives the typographical form of the text according to the
% ``rules'' given in the document class file and in various style files.
\LaTeX 은 텍스트의 논리적 의미론적 구조만을 지시한다는 점에서 다른 조판 시스템과 다르다.
\LaTeX 은 문서 클래스 파일과 여러 스타일 파일에서 주어지는 ``규칙''에 따라 텍스트의 조판 형태를 만들어낸다.

% The most important text unit in \LaTeX{} (and in typography) is the
% \wi{paragraph}.  We call it ``text unit'' because a paragraph is the
% typographical form that should reflect one coherent thought, or one idea.
% You will learn in the following sections how to force line breaks with
% e.g.\ \texttt{\bs\bs}, and paragraph breaks with e.g.\ leaving an empty line
% in the source code.  Therefore, if a new thought begins, a new paragraph
% should begin, and if not, only line breaks should be used.  If in doubt
% about paragraph breaks, think about your text as a conveyor of ideas and
% thoughts.  If you have a paragraph break, but the old thought continues, it
% should be removed.  If some totally new line of thought occurs in the same
% paragraph, then it should be broken.
\LaTeX 에서(일반적으로 타이포그래피에서) 가장 중요한 텍스트 단위는 \wi{문단}[paragraph]이다.
이것을 ``텍스트 단위''라 하는데 그 이유는 문단이 한 가지 일관된 생각이나 개념을 반영하는 타이포그래피상의 형태이기 때문이다.
이어지는 소절에서 예컨대 \texttt{\bs\bs}를 이용하여 줄을 바꾸거나 또는 빈 줄을 두어 문단을 구분하는 방법을 
배우게 될 것이다. 그러므로 새로운 문단이 시작되는 것은 새로운 생각이 시작되었을 때여야 한다. 그런 것이 아니면 
문단 구분이 아니라 \wi{줄나눔}만을 써야 한다.%
\trfnote{%
  `line breaking'의 역어로서 이 번역본에서는 ``줄나눔''과 ``행나눔''을 혼용하였다. `개행(改行)'이라 하기도 한다.
}
문단을 구분하여야 할지 어떨지 잘 모르겠다면 
텍스트가 개념이나 사고의 흐름을 전달하는 매개체라는 관점에서 살펴보라. 
문단이 나누어졌는데 이전의 생각이 이어지고 있다면 그 문단나눔을 제거해야 한다. 
완전히 새로운 사유를 전개하는데 문단이 나누어지지 않았다면 거기서 문단을 나누어야 한다.

% Most people completely underestimate the importance of well-placed
% paragraph breaks.  Many people do not even know what the meaning of
% a paragraph break is, or, especially in \LaTeX, introduce paragraph
% breaks without knowing it.  The latter mistake is especially easy to
% make if equations are used in the text.  Look at the following
% examples, and figure out why sometimes empty lines (paragraph breaks)
% are used before and after the equation, and sometimes not.  (If you
% don't yet understand all commands well enough to understand these
% examples, please read this and the following chapter, and then read
% this section again.)
많은 사람들이 문단나눔을 적절히 하는 것이 얼마나 중요한지를 잘 모르고 있다. 문단나눔의 의미조차
알지 못하는 사람도 많고 특히 \LaTeX{} 사용자 중에 그것이 문단나눔인지조차 알지 못하고 문단을 나누는 사람도 있다.
텍스트 안에 수식이 사용될 때 특히 이런 실수를 자주 저지른다. 
다음 예를 보고 수식을 전후하여 어떨 때 빈 줄(문단나눔)이 들어가고 어떨 때 들어가지 않았는지 이해하자.
(여기 사용된 명령 중에 모르는 것이 있다면 이 장과 다음 장을 읽은 후에 되돌아와서 다시 읽어보기 바란다.)

% \begin{code}
% \begin{verbatim}
% % Example 1
% \ldots when Einstein introduced his formula
% \begin{equation}
%   e = m \cdot c^2 \; ,
% \end{equation}
% which is at the same time the most widely known
% and the least well understood physical formula.


% % Example 2
% \ldots from which follows Kirchhoff's current law:
% \begin{equation}
%   \sum_{k=1}^{n} I_k = 0 \; .
% \end{equation}

% Kirchhoff's voltage law can be derived \ldots


% % Example 3
% \ldots which has several advantages.

% \begin{equation}
%   I_D = I_F - I_R
% \end{equation}
% is the core of a very different transistor model. \ldots
% \end{verbatim}
% \end{code}
\begin{code}
\begin{verbatim}
% Example 1
\ldots when Einstein introduced his formula
\begin{equation}
  e = m \cdot c^2 \; ,
\end{equation}
which is at the same time the most widely known
and the least well understood physical formula.


% Example 2
\ldots from which follows Kirchhoff's current law:
\begin{equation}
  \sum_{k=1}^{n} I_k = 0 \; .
\end{equation}

Kirchhoff's voltage law can be derived \ldots


% Example 3
\ldots which has several advantages.

\begin{equation}
  I_D = I_F - I_R
\end{equation}
is the core of a very different transistor model. \ldots
\end{verbatim}
\end{code}


% The next smaller text unit is a sentence.  In English texts, there is
% a larger space after a period that ends a sentence than after one
% that ends an abbreviation.  \LaTeX{} tries to figure out which one
% you wanted to have.  If \LaTeX{} gets it wrong, you must tell it what
% you want.  This is explained later in this chapter.
문단보다 작은 텍스트 단위는 문장이다. 영문에서 약어 뒤의 마침표보다 문장의 종지를 의미하는 마침표에
더 큰 공백을 둔다. \LaTeX 은 입력된 마침표가 어떤 경우에 해당하는지를 결정하려 하는데 \LaTeX 의 선택이 
적절하지 않으면 글쓴이가 원하는 바를 지시해주어야 한다. 이 장의 뒷부분에서 설명한다.

% The structuring of text even extends to parts of sentences.  Most
% languages have very complicated punctuation rules, but in many
% languages (including German and English), you will get almost every
% comma right if you remember what it represents: a short stop in the
% flow of language.  If you are not sure about where to put a comma,
% read the sentence aloud and take a short breath at every comma.  If
% this feels awkward at some place, delete that comma; if you feel the
% urge to breathe (or make a short stop) at some other place, insert a
% comma.
문장은 더 작은 요소들로 구분된다. 대부분의 언어에서 문장부호 사용법은 매우 복잡하다. 
그렇지만 영어나 독일어 같은 많은 언어에서 쉼표 사용법은 그것이 언어의 흐름상 구분점을 의미한다는 것을
기억하면 거의 실수없이 적용할 수 있다.
쉼표를 어디에 찍어야 할지 모르겠다면 문장을 큰 소리로 읽으면서 쉼표가 있는 위치에서 짧게 숨을 쉬어보자.
그 숨쉬기가 어색하게 느껴진다면 그곳의 쉼표는 지운다. 만약 어딘가 숨을 쉬거나 잠시 휴지(休止)를 두어야겠다고
느낀다면 거기에 쉼표를 삽입한다.

% Finally, the paragraphs of a text should also be structured logically
% at a higher level, by putting them into chapters, sections,
% subsections, and so on.  However, the typographical effect of writing
% e.g.\ \verb|\section{The| \texttt{Structure of Text and Language}\verb|}| is
% so obvious that it is almost self-evident how these high-level
% structures should be used.
마지막으로 문단은 더 상위 수준인 장(chapter), 절(section), 소절(subsection) 등을 구성하도록 논리적 구조에 따라 나열되어야 한다.
예를 들어 \verb|\section{The| \texttt{Structure of Text and Language}\verb|}|라고 적어 넣는 것이 무엇을 의미하는지 그리고 이 어구가 어떻게 배치될지 명령 자체가 이미 다 설명하고 있다.

% \section{Line Breaking and Page Breaking}
\section{줄나눔과 페이지나눔}

% \subsection{Justified Paragraphs}
\subsection{문단 정렬}

% Books are often typeset with each line having the same length.
% \LaTeX{} inserts the necessary \wi{line break}s and spaces between words
% by optimizing the contents of a whole paragraph. If necessary, it
% also hyphenates words that would not fit comfortably on a line.
% How the paragraphs are typeset depends on the document class.
% Normally the first line of a paragraph is indented, and there is no
% additional space between two paragraphs. Refer to section~\ref{parsp}
% for more information.
각 행의 길이가 같도록 \wi{양끝맞춤}(justified)으로 정렬하는 것이 보통 문서의 조판 관행이다.
이를 위하여 \LaTeX 은 전체 문단의 내용을 최적화하여 단어 사이에 \wi{줄나눔}[line breaks]과 
공백을 삽입한다. 필요하다면 한 줄에 잘 들어맞지 않는 단어를 하이픈처리한다.
문단을 조판하는 모양은 문서 클래스를 따른다. 
문단의 첫 줄을 들여쓰기하고 문단과 문단 사이에 추가 간격을 두지 않는 것이 일반적이다.
\pageref{parsp}페이지의 \ref{parsp}절에서 자세히 다룬다.

% In special cases it might be necessary to order \LaTeX{} to break a
% line:
% \begin{lscommand}
% \ci{\bs} or \ci{newline}
% \end{lscommand}
% \noindent starts a new line without starting a new paragraph.
특별한 경우에 \LaTeX 에게 강제로 행을 나누도록 해야 할 때가 있다.
\begin{lscommand}
\verb|\\| 또는 \ci{newline}
\end{lscommand}
\noindent 이 명령은 새로운 문단을 시작하지 않은 상태에서 줄나눔을 행한다.

% \begin{lscommand}
% \ci{\bs*}
% \end{lscommand}
% \noindent additionally prohibits a page break after the forced
% line break.
다음 명령은 강제로 줄나눔을 행하되 페이지나눔은 일어나지 않도록 하는 것이다.
\begin{lscommand}
\verb|\\*|
\end{lscommand}

% \begin{lscommand}
% \ci{newpage}
% \end{lscommand}
% \noindent starts a new page.
새 페이지를 시작하려면 다음 명령을 쓴다.
\begin{lscommand}
\ci{newpage}
\end{lscommand}

% \begin{lscommand}
% \ci{linebreak}\verb|[|\emph{n}\verb|]|,
% \ci{nolinebreak}\verb|[|\emph{n}\verb|]|,
% \ci{pagebreak}\verb|[|\emph{n}\verb|]|,
% \ci{nopagebreak}\verb|[|\emph{n}\verb|]|
% \end{lscommand}
% \noindent suggest places where a break may (or may not) happen. They enable the author to influence their
% actions with the optional argument \emph{n}, which can be set to a number
% between zero and four. By setting \emph{n} to a value below 4, you leave
% \LaTeX{} the option of ignoring your command if the result would look very
% bad. Do not confuse these ``break'' commands with the ``new'' commands. Even
% when you give a ``break'' command, \LaTeX{} still tries to even out the
% right border of the line and the total length of the page, as described in
% the next section; this can lead to unpleasant gaps in your text.
% If you really want to start a ``new line'' or a ``new page'', then use the
% corresponding command. Guess their names!
줄나눔과 페이지나눔에 관련된 명령 몇 가지가 있다.
\begin{lscommand}
\ci{linebreak}\verb|[|\emph{n}\verb|]|,
\ci{nolinebreak}\verb|[|\emph{n}\verb|]|,
\ci{pagebreak}\verb|[|\emph{n}\verb|]|,
\ci{nopagebreak}\verb|[|\emph{n}\verb|]|
\end{lscommand}
\noindent 선택인자 \emph{n}은 0에서 4까지의 값을 가질 수 있는데 이 값을 이용하여 명령이 영향을 미치는 강도를 조절할 수 있다.
\emph{n}을 4 미만으로 설정하는 것은 \LaTeX 이 조판 결과가 좋지 않으면 이 명령을 무시해도 좋다는 의미이다.
``break''와 ``new'' 명령을 혼동하면 안된다. ``break'' 명령이 입력되었을 때 \LaTeX 은 여전히 줄의 오른쪽 끝과 페이지의 아래쪽 끝을
가지런하게 정렬하려고 시도하기 때문에 좋지 않은 조판 결과를 얻게 될 수 있다. 이에 대해 다음 절에서 설명한다.
정말로 ``새로운 줄''이나 ``새로운 페이지''를 시작하려 한다면 ``new'' 명령을 써야한다. 이름이 그렇게 붙여진 이유가 있는 것이다.

% \LaTeX{} always tries to produce the best line breaks possible. If it
% cannot find a way to break the lines in a manner that meets its high
% standards, it lets one line stick out on the right of the paragraph.
% \LaTeX{} then complains (``\wi{overfull hbox}'') while processing the
% input file. This happens most often when \LaTeX{} cannot find a
% suitable place to hyphenate a word.\footnote{Although \LaTeX{} gives
%   you a warning when that happens (\texttt{Overfull \bs{}hbox}) and displays the
%   offending line, such lines are not always easy to find. If you use
%   the option \texttt{draft} in the \ci{documentclass} command, these
%   lines will be marked with a thick black line on the right margin.}
% Instruct \LaTeX{} to lower its standards a little by giving
% the \ci{sloppy} command. It prevents such over-long lines by
% increasing the inter-word spacing---even if the final output is not
% optimal.  In this case a warning (``\wi{underfull hbox}'') is given to
% the user.  In most such cases the result doesn't look very good. The
% command \ci{fussy} brings \LaTeX{} back to its default behaviour.
\LaTeX 은 가능한 한 최선의 줄나눔을 하려고 한다. 만약 \LaTeX 이 설정한 높은 기준을 충족하는
줄나눔이 실패하면 문단에서 그 한 줄이 오른쪽으로 튀어나가도록 조판한다. 그러면서 ``\wi{overfull hbox}''라는 
경고 메시지를 컴파일 과정에 보여준다. \LaTeX 이 단어의 적절한 분철(hyphenate) 위치를 찾지 못했을 때 자주 일어난다.\footnote{%
  \LaTeX 이 \texttt{Overfull \bs{}hbox}가 발생한 줄 번호와 경고메시지를 보여주기는 하지만
  해당 위치를 찾는 것이 항상 쉬운 것은 아니다. \ci{documentclass} 명령의 옵션으로 \texttt{draft}를
  지정하면 이런 일이 발생한 줄의 오른쪽 여백에 두꺼운 검은 선을 그려서 표시해준다.
}
\LaTeX 에게 그 기준을 좀 낮추라고 하려면 \ci{sloppy} 명령을 준다. 이렇게 하면
튀어나가는 줄은 사라지지만 결과가 적절하지 않더라도 단어 사이의 간격을 늘려서 조판하면서 ``\wi{underfull hbox}''
경고를 보여주므로 대부분의 경우에 출력되는 결과는 그다지 좋지 않다. 
\LaTeX 이 원래의 기준으로 조판을 행하도록 되돌리려면 \ci{fussy} 명령을 준다.


% \subsection{Hyphenation} \label{hyph}
\subsection{분철} \label{hyph}

% \LaTeX{} hyphenates words whenever necessary. If the hyphenation
% algorithm does not find the correct hyphenation points,
% remedy the situation by using the following commands to tell \TeX{}
% about the exception.
\LaTeX 은 필요할 때 단어를 \wi{분철}(hyphenate)한다. 분철 알고리즘이 적절한 분철 위치를 찾지 못하면
\TeX{}에게 예외 처리를 지시하는 다음 명령을 써서 문제를 해결한다.

% The command
% \begin{lscommand}
% \ci{hyphenation}\verb|{|\emph{word list}\verb|}|
% \end{lscommand}
% \noindent causes the words listed in the argument to be hyphenated only at
% the points marked by ``\verb|-|''.  The argument of the command should only
% contain words built from normal letters, or rather signs that are considered
% to be normal letters by \LaTeX{}. The hyphenation hints are
% stored for the language that is active when the hyphenation command
% occurs. This means that if you place a hyphenation command into the preamble
% of your document it will influence the English language hyphenation. If you
% place the command after the \verb|\begin{document}| and you are using some
% package for national language support like \pai{polyglossia}, then the hyphenation
% hints will be active in the language activated through \pai{polyglossia}.
\begin{lscommand}
\ci{hyphenation}\verb|{|\emph{word list}\verb|}|
\end{lscommand}
\noindent 이 명령의 인자로 열거되는 단어들은 ``\verb|-|''로 표시된 위치에서만 분철이 이루어진다.
인자로 오는 것은 \LaTeX 이 일반 문자(letter)로 간주하는 문자와 기호를 포함해야 한다. 
분철 위치는 hyphenation 명령이 주어진 위치에서 활성화된 언어 용으로 저장된다. 
즉 hyphenation 명령을 문서의 전처리부에 두었다면 영어 언어 전체의 분철에 영향을 미친다.
이 명령을 \verb|\begin{document}| 이후에 두고 \pai{polyglossia} 같은 다국어 지원 패키지를 활용한다면
\pai{polyglossia}에 의하여 활성화된 언어에서만 동작하게 할 수 있다.

% The example below will allow ``hyphenation'' to be hyphenated as well as
% ``Hyphenation'', and it prevents ``FORTRAN'', ``Fortran'' and ``fortran''
% from being hyphenated at all.  No special characters or symbols are allowed
% in the argument.
다음 예는 ``hyphenation''과 ``Hyphenation''에 동일한 분철 규칙을 적용하고 ``FORTRAN'', ``Fortran'', ``fortran''에 
대해서 분철을 허용하지 않게 한다. 특수문자나 기호는 허용되지 않는다.\trfnote{각 단어 사이에 \texttt{,} 등의 구분 기호가 없음에 주의.}

% Example:
% \begin{code}
% \verb|\hyphenation{FORTRAN Hy-phen-a-tion}|
% \end{code}
\begin{code}
\verb|\hyphenation{FORTRAN Hy-phen-a-tion}|
\end{code}


% The command \ci{-} inserts a discretionary hyphen into a word. This
% also becomes the only point hyphenation is allowed in this word. This
% command is especially useful for words containing special characters
% (e.g.\ accented characters), because \LaTeX{} does not automatically
% hyphenate words containing special characters.
% %\footnote{Unless you are using the new
% %\wi{DC fonts}.}.
\ci{-} 명령은 단어에 임의의 분철 규칙을 삽입한다. 그리고 그 위치가 해당 단어에서 유일하게 분철 가능한 위치가 된다.
이 명령은 특수문자(예: 강세표시 문자)를 포함하고 있는 단어에서 특히 유용한데 특수문자를 포함하는 단어에 대하여 
\LaTeX 이 자동으로 분철하지 않기 때문이다.\trfnote{%
  강세표시 문자가 포함된 단어를 분철하지 못한다는 것은 OT1 인코딩의 폰트를 사용하는 경우에 해당한다.
  현대의 유니코드 텍엔진이나 T1 인코딩 폰트를 사용하는 경우에 이런 문자가 포함된 단어에 대해서도 분철이
  가능하다. 한편 한국어 표기는 분철부호를 쓰지 않으며 모든 음절문자 사이에서(일부 예외를 제외하면) 줄나눔을 할 수 있다.
}

% \begin{example}
% I think this is: su\-per\-cal\-%
% i\-frag\-i\-lis\-tic\-ex\-pi\-%
% al\-i\-do\-cious
% \end{example}
\begin{example}
I think this is: su\-per\-cal\-%
i\-frag\-i\-lis\-tic\-ex\-pi\-%
al\-i\-do\-cious
\end{example}

% Several words can be kept together on one line with the command
% \begin{lscommand}
% \ci{mbox}\verb|{|\emph{text}\verb|}|
% \end{lscommand}
% \noindent It causes its argument to be kept together under all circumstances.
몇 단어를 묶어서 줄나눔을 하지 않도록 하려면 다음과 같이 한다.
\begin{lscommand}
\ci{mbox}\verb|{|\emph{text}\verb|}|
\end{lscommand}
\noindent 인자로 주어진 단어들은 항상 묶인 상태가 된다.

% \begin{example}
% My phone number will change soon.
% It will be \mbox{0116 291 2319}.

% The parameter
% \mbox{\emph{filename}} should
% contain the name of the file.
% \end{example}

\begin{example}
My phone number will change soon.
It will be \mbox{0116 291 2319}.

The parameter
\mbox{\emph{filename}} should
contain the name of the file.
\end{example}

% \ci{fbox} is similar to \ci{mbox}, but in addition there will
% be a visible box drawn around the content.
\ci{fbox}는 \ci{mbox}와 비슷한데 인자로 주어진 단어에 테두리를 그려준다.

% \section{Ready-Made Strings}
\section{미리 정의된 문자열}

% In some of the examples on the previous pages, you have seen
% some very simple \LaTeX{} commands for typesetting special
% text strings:
앞서 몇몇 예제에서 특정 문자열을 식자하는 \LaTeX{} 명령을 이미 보였다. \TeX, \LaTeX 의 
로고를 식자하는 명령이 포함되어 있다.

% \vspace{2ex}

% \noindent
% \begin{tabular}{@{}lll@{}}
% Command&Example&Description\\
% \hline
% \ci{today} & \today   & Current date\\
% \ci{TeX} & \TeX       & Your favorite typesetter\\
% \ci{LaTeX} & \LaTeX   & The Name of the Game\\
% \ci{LaTeXe} & \LaTeXe & The current incarnation\\
% \end{tabular}

\vspace{2ex}

\noindent
\begin{tabular}{@{}lll@{}}
명령어 & 출력 예 & 설명 \\
\hline
\ci{today} & \today   & 시스템의 오늘 날짜\footnotemark \\
\ci{TeX} & \TeX       & 최고의 조판 시스템 \\
\ci{LaTeX} & \LaTeX   & 지금 배우고 있는 것의 이름 \\
\ci{LaTeXe} & \LaTeXe & \LaTeX 의 현재 버전 \\
\end{tabular}

\footnotetext{%
  [역주] 한국어 \TeX 에서 \cs{today} 명령이 만들어내는 문자열은 
  ``2019년 2월 9일''과 같은 모양일 수 있다.
}

% \section{Special Characters and Symbols}
\section{특수문자와 기호}

% \subsection{Quotation Marks}
\subsection{따옴표}

% You should \emph{not} use the \verb|"| for \wi{quotation marks}
% \index{""@\texttt{""}} as you would on a typewriter.  In publishing
% there are special opening and closing quotation marks.  In \LaTeX{},
% use two~\textasciigrave~(grave accent) for opening quotation marks and
% two~\textquotesingle~(vertical quote) for closing quotation marks. For single
% quotes you use just one of each.
% \begin{example}
% ``Please press the `x' key.''
% \end{example}
% Yes I know the rendering is not ideal, it's really a back-tick or grave accent
% (\textasciigrave) for
% opening quotes and vertical quote (\textquotesingle) for closing, despite what the font chosen might suggest.
\wi{따옴표}[quotation marks]를 입력하기 위해 타자기에서처럼 \verb|"|를 사용해서는 \emph{안 된다}.
출판물에서 사용하는 여는 따옴표와 닫는 따옴표는 모양이 다르다. \LaTeX 은 두 개의 \textasciigrave (grave accent)로 여는 따옴표를,
두 개의 \textquotesingle (vertical quote)로 닫는 따옴표를 표시한다. 작은따옴표는 이것을 한 번씩 사용한다.
\begin{example}
``Please press the `x' key.''
\end{example}
편집기 상에서 (폰트에 따라 다르겠지만) 여는 따옴표가 back-tick이나 grave accent(\textasciigrave)이고 닫는 따옴표가 quote(\textquotesingle)인 것이
마음에 들지 않을 수도 있지만 그렇게 입력해야 하는 것으로 기억하자.%
\trfnote{%
  \TeX{}Shop이나 \TeX{}works 같은 \LaTeX{} 전용 편집기는 사용자의 따옴표 입력 편의를 위해 
  키보드의 \texttt{"}를 연달아 입력해도 지능적으로 \texttt{``} \texttt{''}로 바꾸어주는 기능이 있다.
  \XeLaTeX 을 위해 유니코드 따옴표 \texttt{“} \texttt{”}로 바꾸어주기도 한다.
  그러나 소스에 키보드 따옴표 문자 \texttt{"}가 그대로 입력되어서는 안 된다는 것은 유효하다.
}


% \subsection{Dashes and Hyphens}
\subsection{대시와 하이픈}

% \LaTeX{} knows four kinds of \wi{dash}es. Access three of
% them with different number of consecutive dashes. The fourth sign
% is actually not a dash at all---it is the mathematical minus sign: \index{-}
% \index{--} \index{---} \index{-@$-$} \index{mathematical!minus}
\LaTeX 에 네 가지 \wi{대시}[dash]가 있다. 세 가지는 잇대어 입력하는 대시의 개수에 따라 달라진다.
네 번째 것은 사실 대시가 아니라 수학 부호이다. \index{-}\index{--}\index{---}\index{-@$-$} \index{수학 뺄셈 부호}

% \begin{example}
% daughter-in-law, X-rated\\
% pages 13--67\\
% yes---or no? \\
% $0$, $1$ and $-1$
% \end{example}
% The names for these dashes are:
% `-' \wi{hyphen}, `--' \wi{en-dash}, `---' \wi{em-dash} and
% `$-$' \wi{minus sign}.
\begin{example}
daughter-in-law, X-rated\\
pages 13--67\\
yes---or no? \\
$0$, $1$ and $-1$
\end{example}
\noindent 각각 `-' \wi{하이픈}[hyphen], `--' \wi{엔대시}[en-dash], `---' \wi{엠대시}[em-dash], $-$ \wi{뺄셈 부호}[minus sign]라고 부른다.\trfnote{%
  한글 맞춤법에 이 부호와 유사한 모양의 `줄표'와 `붙임표'가 있다.
  길이에 따른 대시의 종류 구분은 없으며 사용법에도 차이가 있다.
  예를 들어 영어 문장에서는 범위를 나타내기 위하여 엔대시를 쓰지만
  한글 맞춤법에서는 이 자리에 `물결표($\sim$)'를 쓰도록 규정하고 있다.
  붙임표는 분철(하이프네이션)과는 아무 관련 없고 이어지는 내용을 묶어 열거할 때 사용하는 것이다. 엠대시를 우리 글의 줄표(어구를 나누거나 강조 생략하기 위해 쓰는 부호)를 쓸 자리에 쓸 수 있다.
}


% \subsection{Tilde ($\sim$)}
% \index{URL link}\index{tilde}
\subsection{틸데 (\texorpdfstring{$\sim$}{~})} \index{URL link}\index{tilde}

% A character often seen in web addresses is the tilde. To generate
% this in \LaTeX{} use \verb|\~{}| but the result (\~{}) is not really
% what you want. Try this instead:
웹 주소에서 흔히 보이는 문자 \verb|~|가 틸데이다. \LaTeX 에서 \verb|\~{}|로 입력하면
\~{}를 출력해주지만 원하던 것이 아닐 수 있다. 다음 예를 보라.\trfnote{%
  웹 주소를 문서에 적어넣을 적에 틸데 문자를 그대로 사용할 수 있게 하는 \pai{url} 또는 \pai{hyperref}
  패키지의 \ci{url} 명령을 쓰는 것이 좋다. 
}

% \begin{example}
% http://www.rich.edu/\~{}bush \\
% http://www.clever.edu/$\sim$demo
% \end{example}
\begin{example}
http://www.rich.edu/\~{}bush \\
http://www.clever.edu/$\sim$demo
\end{example}

% \subsection{Slash (/)}
% \index{Slash}
\subsection{슬래시 (/)} \index{slash}

% In order to typeset a slash between two words, one can simply type e.g.\
% \texttt{read/write}, but this makes \LaTeX{} treat the two words as one.
% Hyphenation is disabled for these two words, so there may be `overfull'
% errors.  To overcome this, use \ci{slash}.  For example type
% `\verb|read\slash write|' which allows hyphenation.  But normal `\texttt{/}'
% character may be still used for ratios or units, e.g.\ \texttt{5 MB/s}.
두 단어 사이에 슬래시를 넣으려면 간단히 \texttt{read/write}와 같이 입력할 수 있다.
그렇지만 이렇게 하면 이 전체를 하나의 단어로 인식하게 된다. 단어 내부에서 분철이 일어나지 않기 
때문에 `overfull' 경고를 만날 수 있다. \ci{slash} 명령을 사용하여 예컨대 \verb|read\slash write|와 
같이 입력하면 분철이 허용되게 된다. 그러나 `\texttt{/}' 문자는 분수나 단위를 나타낼 때는 
그대로 사용해야 한다.
\texttt{5 MB/s}.

% \subsection{Degree Symbol \texorpdfstring{($\circ$)}{}}
\subsection{도 기호 \texorpdfstring{($\circ$)}{}}

% Printing the \wi{degree symbol} in pure \LaTeX{}.
\wi{도 기호}[degree symbol]를 입력하는 방법을 다음 예에서 볼 수 있다.

% \begin{example}
% It's $-30\,^{\circ}\mathrm{C}$.
% I will soon start to
% super-conduct.
% \end{example}
\begin{example}
It's $-30\,^{\circ}\mathrm{C}$.
I will soon start to
super-conduct.
\end{example}

% The \pai{textcomp} package makes the degree symbol also available as
% \ci{textdegree} or in combination with the C by using the \ci{textcelsius}.
또는, \ci{textdegree} 명령으로 입력하는 것도 가능하다. \ci{textcelsius} 명령은 도 기호 뒤에 C를 붙여준다.

% \begin{example}
% 30 \textcelsius{} is
% 86 \textdegree{}F.
% \end{example}
\begin{example}
30 \textcelsius{} is
86 \textdegree{}F.
\end{example}
\noindent 이 경우에, 만약 \XeLaTeX 이나 Lua\LaTeX 이 아닌 legacy \TeX (pdf\LaTeX)으로 컴파일하고 있다면 \pai{textcomp}
패키지를 usepackage 하여야 한다.

% \subsection{The Euro Currency Symbol \texorpdfstring{(\officialeuro)}{}}
\subsection{유로 통화 기호(\texorpdfstring{{\texteuro}}{})}

% When writing about money these days, you need the Euro symbol. Many current
% fonts contain a Euro symbol. After loading the \pai{textcomp} package in the preamble of your document
% \begin{lscommand}
% \ci{usepackage}\verb|{textcomp}|
% \end{lscommand}
% use the command
% \begin{lscommand}
% \ci{texteuro}
% \end{lscommand}
% to access it.
이제 금융 관련한 글을 쓰려면 유로화 기호가 필요한 시대가 되었다.
오늘날 많은 폰트들이 이 기호를 이미 포함하고 있으므로 다음과 같이 하는 것으로 충분하다.
\begin{example}
  \texteuro
\end{example}
\noindent 레거시 텍 엔진(pdf\LaTeX)을 쓰고 있다면 \pai{textcomp} 패키지가 필요하다.

이 번역본의 대본이 된 영문판 lshort에는 유로화 기호에 대한 다양한 표현 방법을 소개하고 있으나 
일부는 \XeLaTeX 과 같은 현대적 \TeX\ 엔진에서 불필요하거나 호환되지 않는 것이고 우리 실정에
꼭 들어맞는다고 할 수 없어서 번역본에서는 해당 부분을 제외하였다.

참고로, \wi{원화 기호}와 \wi{엔화 기호}는 다음과 같이 표현할 수 있다.
(마찬가지로 \pai{textcomp} 패키지를 요구할 수 있다.)
\begin{example}
  \textwon, \textyen
\end{example}

% If your font does not provide its own Euro symbol or if you do not like the
% font's Euro symbol, you have two more choices:

% First the \pai{eurosym} package. It provides the official Euro symbol:
% \begin{lscommand}
% \ci{usepackage}\verb|[|\emph{official}\verb|]{eurosym}|
% \end{lscommand}
% If you prefer a Euro symbol that matches your font, use the option
% \texttt{gen} in place of the \texttt{official} option.

% %If the Adobe Eurofonts are installed on your system (they are available for
% %free from \url{ftp://ftp.adobe.com/pub/adobe/type/win/all}) you can use
% %either the package \pai{europs} and the command \ci{EUR} (for a Euro symbol
% %that matches the current font).
% % does not work
% % or the package
% % \pai{eurosans} and the command \ci{euro} (for the ``official Euro'').

% %The \pai{marvosym} package also provides many different symbols, including a
% %Euro, under the name \ci{EURtm}. Its disadvantage is that it does not provide
% %slanted and bold variants of the Euro symbol.

% \begin{table}[!htbp]
% \caption{A bag full of Euro symbols} \label{eurosymb}
% \begin{lined}{10cm}
% \begin{tabular}{llccc}
% LM+textcomp  &\verb+\texteuro+ & \huge\texteuro &\huge\sffamily\texteuro
%                                                 &\huge\ttfamily\texteuro\\
% eurosym      &\verb+\euro+ & \huge\officialeuro &\huge\sffamily\officialeuro
%                                                 &\huge\ttfamily\officialeuro\\
% $[$gen$]$eurosym &\verb+\euro+ & \huge\geneuro  &\huge\sffamily\geneuro
%                                                 &\huge\ttfamily\geneuro\\
% %europs       &\verb+\EUR + & \huge\EURtm        &\huge\EURhv
% %                                                &\huge\EURcr\\
% %eurosans     &\verb+\euro+ & \huge\EUROSANS  &\huge\sffamily\EUROSANS
% %                                             & \huge\ttfamily\EUROSANS \\
% %marvosym     &\verb+\EURtm+  & \huge\mvchr101  &\huge\mvchr101
% %                                               &\huge\mvchr101
% \end{tabular}
% \medskip
% \end{lined}
% \end{table}

% \subsection{Ellipsis (\texorpdfstring{\ldots}{...})}
\subsection{줄임표 (\texorpdfstring{\ldots}{...})}

% On a typewriter, a \wi{comma} or a \wi{period} takes the same amount of
% space as any other letter. In book printing, these characters occupy
% only a little space and are set very close to the preceding letter.
% Therefore, entering `\wi{ellipsis}' by just typing three
% dots would produce the wrong result. Instead, there is a special
% command for these dots. It is called
타자기에서는 \wi{쉼표}나 \wi{마침표}가 다른 글자와 같은 폭을 차지하지만
인쇄된 서적에서는 이 글자들이 그 앞 글자에 아주 가깝게 붙는다. 그러므로 `\wi{줄임표}[ellipsis]'를 
나타내기 위하여 점 세 개로 표현하면 제대로 표현되지 않는다. 이를 위한 별도의
명령이 있다.\trfnote{%
  한글 맞춤법에서 규정하고 있는 `줄임표'는 여섯 개의 가운뎃점으로 표현한다. 
  이를 위해서 \pai{kotex} 패키지가 재정의한 \ci{ldots}를 두 번 써야 한다\hdots\hdots.
}

% \begin{lscommand}
% \ci{ldots} (low dots)
% \end{lscommand}
% \index{...@\ldots}


% \begin{example}
% Not like this ... but like this:\\
% New York, Tokyo, Budapest, \ldots
% \end{example}

\begin{lscommand}
\ci{ldots} (low dots)
\end{lscommand}
\index{...@\ldots}

\begin{example}
Not like this ... but like this:\\
New York, Tokyo, Budapest\ldots
\end{example}

% \subsection{Ligatures}
\subsection{합자}

% Some letter combinations are typeset not just by setting the
% different letters one after the other, but by actually using special
% symbols.
% \begin{code}
% {\large ff fi fl ffi\ldots}\quad
% instead of\quad {\large f{}f f{}i f{}l f{}f{}i \ldots}
% \end{code}
% These so-called \wi{ligature}s can be prohibited by inserting an \ci{mbox}\verb|{}|
% between the two letters in question. This might be necessary with
% words built from two words.
라틴문자의 인쇄에 있어 낱글자를 각각 식자하지 않고 몇 글자를 묶어서 하나의 활자로 식자하는 경우가 있다.
이것을 합자(ligature)라고 한다. 실제로는 특수기호를 사용하여 식자하는 것이다.
\begin{code}
{\large ff fi fl ffi\ldots}\quad
instead of\quad {\large f\mbox{}f f\mbox{}i f\mbox{}l f\mbox{}f\mbox{}i \ldots}
\end{code}
합자로 식자하지 않고 각 글자를 낱낱이 찍게 하려면 \ci{mbox}\verb|{}|를 합쳐지는 글자 사이에 넣는 방법이 있다.
두 단어가 합쳐져 이루어진 단어의 경계 위치에서 필요할 수 있다.

% \begin{example}
% \Large Not shelfful\\
% but shelf\mbox{}ful
% \end{example}
\begin{example}
\Large Not shelfful\\
but shelf\mbox{}ful
\end{example}

% \subsection{Accents and Special Characters}
\subsection{액센트와 특수 문자}

% \LaTeX{} supports the use of \wi{accent}s and \wi{special character}s
% from many languages. Table~\ref{accents} shows all sorts of accents
% being applied to the letter o. Naturally other letters work too.
\LaTeX 은 여러 언어의 \wi{액센트}[accent]와 \wi{특수 문자}[special character]를 지원한다.
표~\ref{accents}\는 o 문자에 다양한 액센트를 붙인 예를 보여준다. 다른 글자에도 마찬가지로 적용 가능하다.\trfnote{%
  \XeTeX 이나 Lua\TeX 과 같은 유니코드 텍 엔진을 사용할 적에 액센트 붙은 문자는 유니코드 문자를 그대로 입력해도 잘 처리한다.
  ò ó ô õ ö ø.
}

% To place an accent on top of an i or a j, its dots have to be
% removed. This is accomplished by typing \verb|\i| and \verb|\j|.
i와 j 위에 액센트를 두려 할 때 점을 제거해야 할 필요가 있다. \verb|\i|, \verb|\j|와 같이 입력하면 된다.

% \begin{example}
% H\^otel, na\"\i ve, \'el\`eve,\\
% sm\o rrebr\o d, !`Se\~norita!,\\
% Sch\"onbrunner Schlo\ss{}
% Stra\ss e
% \end{example}

% \begin{table}[!hbp]
% \caption{Accents and Special Characters.} \label{accents}
% \begin{lined}{10cm}
% \begin{tabular}{*4{cl}}
% \mstA{\`o} & \mstA{\'o} & \mstA{\^o} & \mstA{\~o} \\
% \mstA{\=o} & \mstA{\.o} & \mstA{\"o} & \mstB{\c}{c}\\[6pt]
% \mstB{\u}{o} & \mstB{\v}{o} & \mstB{\H}{o} & \mstB{\c}{o} \\
% \mstB{\d}{o} & \mstB{\b}{o} & \mstB{\t}{oo} \\[6pt]
% \mstA{\oe}  &  \mstA{\OE} & \mstA{\ae} & \mstA{\AE} \\
% \mstA{\aa} &  \mstA{\AA} \\[6pt]
% \mstA{\o}  & \mstA{\O} & \mstA{\l} & \mstA{\L} \\
% \mstA{\i}  & \mstA{\j} & !` & \verb|!`| & ?` & \verb|?`|
% \end{tabular}
% \index{dotless \i{} and \j}\index{Scandinavian letters}
% \index{ae@\ae}\index{umlaut}\index{grave}\index{acute}
% \index{oe@\oe}\index{aa@\aa}

% \bigskip
% \end{lined}
% \end{table}

\begin{example}
H\^otel, na\"\i ve, \'el\`eve,\\
sm\o rrebr\o d, !`Se\~norita!,\\
Sch\"onbrunner Schlo\ss{}
Stra\ss e
\end{example}

\begin{table}[!hbp]
\caption{액센트와 특수 문자} \label{accents}
\begin{lined}{10cm}
\begin{tabular}{*4{cl}}
\mstA{\`o} & \mstA{\'o} & \mstA{\^o} & \mstA{\~o} \\
\mstA{\=o} & \mstA{\.o} & \mstA{\"o} & \mstB{\c}{c}\\[6pt]
\mstB{\u}{o} & \mstB{\v}{o} & \mstB{\H}{o} & \mstB{\c}{o} \\
\mstB{\d}{o} & \mstB{\b}{o} & \mstB{\t}{oo} \\[6pt]
\mstA{\oe}  &  \mstA{\OE} & \mstA{\ae} & \mstA{\AE} \\
\mstA{\aa} &  \mstA{\AA} \\[6pt]
\mstA{\o}  & \mstA{\O} & \mstA{\l} & \mstA{\L} \\
\mstA{\i}  & \mstA{\j} & !` & \verb|!`| & ?` & \verb|?`|
\end{tabular}
\index{dotless \i{} and \j}\index{Scandinavian letters}
\index{ae@\ae}\index{umlaut}\index{grave}\index{acute}
\index{oe@\oe}\index{aa@\aa}

\bigskip
\end{lined}
\end{table}

% \section{International Language Support}
% \secby{Axel Kielhorn}{A.Kielhorn@web.de}%
\section{다국어 지원}
\secby{Axel Kielhorn}{A.Kielhorn@web.de}%
% \index{international} When you write documents in \wi{language}s
% other than English, there are three areas where \LaTeX{} has to be
% configured appropriately:
\LaTeX 으로 영어가 아닌 다른 언어로 문서를 작성하는 데 있어 다음 세 가지가 마련되어 있어야 한다.\index{다국어}\index{international}

% \begin{enumerate}
% \item All automatically generated text strings\footnote{Table of
%     Contents, List of Figures, \ldots} have to be adapted to the new
%   language.
% \item \LaTeX{} needs to know the hyphenation rules for the current language.
% \item Language specific typographic rules. In French for example, there is a
%   mandatory space before each colon character (:).
% \end{enumerate}
\begin{enumerate} \firmlist
  \item 자동 생성되는 문자열\footnote{목차(Table of Contents), 그림 목차(List of Figures), \ldots.}이 해당 언어에 알맞게 적용되어야 한다.
  \item 해당 언어의 분철 규칙을 \LaTeX 이 알 수 있도록 해야 한다.
  \item 언어와 문자마다 나름의 타이포그래피 규칙이 있다. 예컨대 프랑스어에서는 각 콜론(:) 앞에 공백을 꼭 두는 것이 관행이다.
\end{enumerate}

% Also entering text in your language of choice might be a bit cumbersome using all the
% commands from figure~\ref{accents}. To overcome this problem, until recently you had to delve
% deep into the abyss of language specific encodings both for input as well as fonts. These days,
% with modern \TeX{} engines speaking UTF-8 natively, these problems have relaxed considerably.
또한 자신의 언어로 된 텍스트를 입력함에 있어서 표~\ref{accents}에서 보인 명령들을 일일이 타이핑하는 것은 좀 귀찮은 일이다.
이런 문제를 극복하기 위해서 최근까지 입력 인코딩이며 폰트 인코딩이라는 골치아픈 영역을 알아야 했다.
그러나 오늘날 현대적 \TeX{} 엔진은 자연스럽게 UTF-8을 읽고 쓸 수 있게 되었으며 문제점들은 상당한 정도로 완화되었다.

% The package \pai{polyglossia}\cite{polyglossia} is a replacement for
% venerable \pai{babel} package. It takes care of the hyphenation patterns and automatically
% generated text strings in your documents.
\pai{polyglossia}\cite{polyglossia}는 \pai{babel} 패키지를 대체하는 패키지로서 각 언어의 분철 패턴과 자동 생성 문자열을 
처리해준다.

% The package \pai{fontspec}\cite{fontspec} handles font loading for
% \hologo{XeLaTeX} and \hologo{LuaTeX}. The default font is Latin Modern
% Roman.
\pai{fontspec}\cite{fontspec}은 \hologo{XeLaTeX}과 \hologo{LuaTeX}에서 폰트 사용을 제어한다.
기본 폰트는 Latin Modern Roman이다.

% \subsection{Polyglossia Usage}
\subsection{Polyglossia 사용법}

% Depending on the \TeX{} engine you use slighly different commands are
% necessary in the preamble of your document to properly enable multilingual
% processing. Figure~\ref{allinone} on page~\pageref{allinone} shows a sample preamble that takes care of all the necessary settings.
다중언어를 적절히 처리하게 하기 위해 문서의 전처리부에 적어야 하는 명령이 \TeX\ 엔진에 따라 조금씩 다르다.
\pageref{allinone}페이지의 그림~\ref{allinone}\은 전처리부에서의 언어 설정에 관한 예시이다.

% \begin{figure}[!bp]
% \begin{lined}{10cm}
% \begin{verbatim}
% \usepackage{iftex}
% \ifXeTeX
%    \usepackage{fontspec}
% \else
%    \usepackage{luatextra}
% \fi
% \defaultfontfeatures{Ligatures=TeX}
% \usepackage{polyglossia}
% \end{verbatim}
% \end{lined}
% \caption[All in one preamble]{All in one preamble that takes care of \hologo{LuaLaTeX} and \hologo{XeLaTeX}} \label{allinone}
% \end{figure}
\begin{figure}[!tp]
\begin{lined}{10cm}
\begin{verbatim}
\usepackage{iftex}
\ifXeTeX
   \usepackage{fontspec}
\else
   \usepackage{luatextra}
\fi
\defaultfontfeatures{Ligatures=TeX}
\usepackage{polyglossia}
\end{verbatim}
\end{lined}
\caption[전처리부의 설정]{\hologo{LuaLaTeX}과 \hologo{XeLaTeX}을 위한 전처리부 다국어 처리 일괄 설정} \label{allinone}
\end{figure}

% So far there has been no advantage to using a Unicode \hologo{TeX} engine.
% This changes when we leave the Latin script and move to a more interesting
% language like Greek or Russian.  With a Unicode based system, you can
% simply\footnote{For small values of simple.} enter the native characters in your
% editor and \hologo{TeX} will understand them.
그 동안은 유니코드 \TeX{} 엔진을 사용해서 얻을 이득이 별로 없었다. 이 상황은 라틴 문자를 벗어나서 
그리스어나 러시아어와 같은 흥미로운 언어를 만나면서 바뀌게 되었다.
유니코드에 기반을 둔 시스템을 사용하면 에디터에 고유 문자를 손쉽게\footnote{손쉽다는 말의 최소한도의 의미에서} 입력해넣을 
수 있다. 그리고 \TeX 이 그것을 이해한다.

% Writing in different languages is easy, just specify the languages in the
% preamble. This example uses the \pai{csquotes} package which generates
% the right kind of quotes according to the language you are writing in. Note
% that it needs to be loaded \emph{before} loading the language support.
다국어로 글쓰는 것은 간단하다. 전처리부에 해당 언어를 특정하기만 하면 된다. 다음 예에서 \pai{csquotes} 패키지를
사용한 것을 볼 수 있는데 사용 중인 언어에 따라 적절한 유형의 인용부호를 생성해준다.
이 패키지를 언어 설정보다 \emph{앞에} 두어야 한다는 점을 기억하자.

% \begin{lscommand}
% \verb|\usepackage[autostyle=true]{csquotes}|\\
% \verb|\setdefaultlanguage{english}|\\
% \verb|\setotherlanguage{german}|
% \end{lscommand}
\begin{lscommand}
\verb|\usepackage[autostyle=true]{csquotes}|\\
\verb|\setdefaultlanguage{english}|\\
\verb|\setotherlanguage{german}|
\end{lscommand}
% %
% To write a paragraph in German, you can use the German environment:
이제 독일어 문단은 german 환경으로 쓸 수 있다.

% \begin{example}
% English text.
% \begin{german}
% Deutscher \enquote{Text}.
% \end{german}
% More English \enquote{text}.
% \end{example}
\begin{example}
English text.
\begin{german}
Deutscher \enquote{Text}.
\end{german}
More English \enquote{text}.
\end{example}

% If you just need a word in a foreign language you can use the
% \verb|\text|\emph{language} command:
다른 언어로 한 단어 정도를 적으려면 \verb|\text|\emph{language} 명령을 쓸 수 있다.

% \begin{example}
% Did you know that
% \textgerman{Gesundheit} is
% actually a German word.
% \end{example}
\begin{example}
Did you know that
\textgerman{Gesundheit} is
actually a German word.
\end{example}

% This may look unnecessary since the only advantage is a correct hyphenation,
% but when the second language is a little bit more exotic it will be worth
% the effort.
이 예의 경우는 별 필요없는 일로 보일 수 있지만 그래도 적절한 분철을 얻는다는 장점이 한 가지 있다.
그러나 제2의 언어가 약간 특이한 것이라면 노력을 들일 필요가 없지 않을 것이다.

% Sometimes the font used in the main document does not contain glyphs that
% are required in the second language. Latin Modern for example does not contain
% Cyrillic letters. The solution is to define a font that will be used for
% that language. Whenever a new language is activated, \pai{polyglossia} will
% first check whether a font has been defined for that language. 
% If you are happy with the computer modern font, 
% you may want to try the \enquote{Computer Modern Unicode} font by adding the following commands to the preamble of your document.
문서에서 사용된 폰트에 제2 언어에서 요구하는 글리프가 빠져 있을 수가 있다.
Latin Modern 폰트를 예로 들자면 키릴 문자를 포함하고 있지 않다.
해결책은 해당 언어에 적합한 폰트를 하나 정의하는 일이다. 
\pai{polyglossia} 패키지는 새로운 언어가 활성화되면 먼저 그 언어 용으로 정의된 폰트가 있는지를 체크한다.
computer modern 폰트가 마음에 든다면 다음 내용을 문서 전처리부에 두어서 \enquote{Computer Modern Unicode} 폰트를 사용하게 할 수 있다.

% \medskip\noindent For \hologo{LuaLaTeX} it is pretty simple
% \begin{verbatim}
% \setmainfont{CMU Serif}
% \setsansfont{CMU Sans Serif}
% \setmonofont{CMU Typewriter Text}
% \end{verbatim}
% \noindent For \hologo{XeLaTeX} you have to be a bit more explicit:
% \begin{verbatim}
% \setmainfont{cmun}[
%    Extension=.otf,UprightFont=*rm,ItalicFont=*ti,
%    BoldFont=*bx,BoldItalicFont=*bi,
%  ]
%  \setsansfont{cmun}[
%    Extension=.otf,UprightFont=*ss,ItalicFont=*si,
%    BoldFont=*sx,BoldItalicFont=*so,
%  ]
%  \setmonofont{cmun}[
%    Extension=.otf,UprightFont=*btl,ItalicFont=*bto,
%    BoldFont=*tb,BoldItalicFont=*tx,
%  ]
% \end{verbatim}
\medskip\noindent \hologo{LuaLaTeX}은 간단하다.
\begin{verbatim}
\setmainfont{CMU Serif}
\setsansfont{CMU Sans Serif}
\setmonofont{CMU Typewriter Text}
\end{verbatim}
\noindent \hologo{XeLaTeX}을 위해서는 조금 더 구체적으로 설정한다.
\begin{verbatim}
\setmainfont{cmun}[
   Extension=.otf,UprightFont=*rm,ItalicFont=*ti,
   BoldFont=*bx,BoldItalicFont=*bi,
 ]
 \setsansfont{cmun}[
   Extension=.otf,UprightFont=*ss,ItalicFont=*si,
   BoldFont=*sx,BoldItalicFont=*so,
 ]
 \setmonofont{cmun}[
   Extension=.otf,UprightFont=*btl,ItalicFont=*bto,
   BoldFont=*tb,BoldItalicFont=*tx,
 ]
\end{verbatim}

% With the appropriate fonts loaded, you can now write:
폰트 설정이 적절하게 이루어지면 다음과 같이 할 수 있다.

% \begin{example}
% \textrussian{Правда} is
% a russian newspaper.
% \textgreek{ἀλήθεια} is truth
% or disclosure in philosophy
% \end{example}
{\setmonofont{cmuntt.otf}
\begin{example}
\textrussian{Правда} is
a russian newspaper.
\textgreek{ἀλήθεια} is truth
or disclosure in philosophy
\end{example}
}

% The package \pai{xgreek}\index{Greek}\cite{xgreek} offers support for
% writing either ancient or modern (monotonic or polytonic) greek.
\pai{xgreek} 패키지\index{Greek}\cite{xgreek}는 고전 그리스어와 현대 그리스어(monotonic 또는 polytonic)
처리를 지원한다.

% \subsubsection{Right to Left (RTL) languages.}
\subsubsection{오른쪽에서 왼쪽으로 쓰는(RTL) 언어}

% Some languages are written left to right, others are written right to
% left(RTL). \pai{polyglossia} needs the \pai{bidi}\cite{bidi}
% package\footnote{\texttt{bidi} does not support \hologo{LuaTeX}.} in order
% to support RTL languages. The \pai{bidi} package should be the last package
% you load, even after \pai{hyperref} which is usually the last package.
% (Since \pai{polyglossia} loads \pai{bidi} this means that \pai{polyglossia}
% should be the last package loaded.)
왼쪽에서 오른쪽으로 쓰는 것이 보통이지만 오른쪽에서 왼쪽으로 쓰는(RTL) 언어가 있다.
RTL 언어를 식자하기 위하여 \pai{polyglossia} 패키지는 \pai{bidi} 패키지를 요구한다. (\pai{bidi}는 \hologo{LuaTeX}을 지원하지 않는다.)
\pai{bidi} 패키지는 보통 마지막에 로드하는 \pai{hyperref}보다도 뒤, 마지막에 위치하여야 한다. 
(\pai{polyglossia}가 \pai{bidi}를 로드하기 때문에 \pai{polyglossia}도 마지막으로 로드하는 패키지가 되어야 한다.)

% The package \pai{xepersian}\index{Persian}\cite{xepersian} offers support
% for the Persian language. It supplies Persian \LaTeX-commands that allows
% you to enter commands like \verb|\section| in Persian, which makes this
% really attractive to native speakers. \pai{xepersian} is the only package
% that supports kashida\index{kashida} with \hologo{XeLaTeX}. A package for
% Syriac which uses a similar algorithm is under development.
\pai{xepersian}\index{Persian}\cite{xepersian} 패키지는 페르시아어를 지원한다.
\verb|\section|과 같은 \LaTeX\ 명령을 페르시아어로 입력할 수 있도록 하기 때문에
원어 사용자에게 매력적이다. 이 패키지는 \hologo{XeLaTeX}을 통하여 카시다\index{kashida}%
\trfnote{%
  아랍어 알파벳의 문자 사이 연결부를 길게 늘려 식자하는 모양을 말한다.
}%
를 식자할 수 있는 유일한 패키지이다. 
비슷한 알고리즘을 사용하는 시리아어 패키지가 개발 중에 있다.

% The IranNastaliq font provided by the SCICT\footnote{Supreme Council of
% Information and Communication Technology} is available at their website
% \url{http://www.scict.ir/Portal/Home/Default.aspx}.
SCICT\footnote{Supreme Council of Information and Communication Technology}이 제공하는
IranNastaliq 폰트는 웹사이트 \url{http://www.scict.ir/Postal/Home/Default.aspx}에서 내려받을 수 있다.

% The \pai{arabxetex}\cite{arabxetex} package supports several languages with
% an Arabic script:
\pai{arabxetex}\cite{arabxetex} 패키지는 아랍 문자를 사용하는 여러 언어를 지원한다.

% \begin{itemize}
% \item arab (Arabic)\index{Arabic}
% \item persian\index{Persian}
% \item urdu\index{Urdu}
% \item sindhi\index{Sindhi}
% \item pashto\index{Pashto}
% \item ottoman (turk)\index{Ottoman}\index{Turkish}
% \item kurdish\index{Kurdish}
% \item kashmiri\index{Kashmiri}
% \item malay (jawi)\index{Malay}\index{Jawi}
% \item uighur\index{Uighur}
% \end{itemize}
\begin{itemize} \tightlist
\item arab (Arabic)\index{Arabic}
\item persian\index{Persian}
\item urdu\index{Urdu}
\item sindhi\index{Sindhi}
\item pashto\index{Pashto}
\item ottoman (turk)\index{Ottoman}\index{Turkish}
\item kurdish\index{Kurdish}
\item kashmiri\index{Kashmiri}
\item malay (jawi)\index{Malay}\index{Jawi}
\item uighur\index{Uighur}
\end{itemize}

% It offers a font mapping that enables \hologo{XeLaTeX} to process input
% using the Arab\TeX\ ASCII transcription.
이 패키지는 Arab\TeX\ 아스키 전사법으로 입력한 소스를 \hologo{XeLaTeX}으로 처리하게 하는 폰트 매핑을 제공한다.

% Fonts that support several Arabic laguages are offered by the
% IRMUG\footnote{Iranian Mac User Group} at
% \url{http://wiki.irmug.org/index.php/X_Series_2}.
여러 종류의 아랍어를 지원하는 폰트를 IRMUG\footnote{Iranian Mac User Group}이 제공한다.
\url{http://wiki.irmug.org/index.php/X_Series_2}.

% There is no package available for Hebrew\index{Hebrew} because none is
% needed. The Hebrew support in \pai{polyglossia} should be sufficient. But
% you do need a suitable font with real Unicode Hebrew. SBL Hebrew is free for
% non-commercial use and available at
% \url{http://www.sbl-site.org/educational/biblicalfonts.aspx}. Another font
% available under the Open Font License is Ezra SIL, available at
% \url{http://www.sil.org/computing/catalog/show_software.asp?id=76}.
히브리어\index{Hebrew}를 위한 패키지는 없다. \pai{polyglossia}의 히브리어 지원이 충분해서
불필요하기 때문이다. 그러나 유니코드 히브리어 폰트가 필요하다. SBL Hebrew는 비상업적 이용이 가능한 
자유폰트이며 \url{http://www.sbl-site.org/educational/biblicalfonts.aspx}에서 얻을 수 있다.
Open Font 라이선스로 이용가능한 Ezra SIL 폰트도 있다. \url{http://www.sil.org/computing/catalog/show_software.asp?id=76}에서
얻을 수 있다.

% Remember to select the correct script:
script를 잘 지정해야 한다는 점을 기억하자.

% \begin{lscommand}
% \verb|\newfontfamily\hebrewfont[Script=Hebrew]{SBL Hebrew}| \\
% \verb|\newfontfamily\hebrewfont[Script=Hebrew]{Ezra SIL}|
% \end{lscommand}
\begin{lscommand}
\verb|\newfontfamily\hebrewfont[Script=Hebrew]{SBL Hebrew}| \\
\verb|\newfontfamily\hebrewfont[Script=Hebrew]{Ezra SIL}|
\end{lscommand}


% \subsubsection{Chinese, Japanese and Korean (CJK)}
% \index{Chinese}\index{Japanese}\index{Korean}
\subsubsection{중국어, 일본어, 한국어 (CJK)}
\index{Chinese}\index{Japanese}\index{Korean}\index{중국어}\index{일본어}\index{한국어}

% The package \pai{xeCJK}\cite{xecjk} takes care of font selection and
% punctuation for these languages.
\pai{xeCJK}\cite{xecjk} 패키지가 이 언어들에 대한 폰트 선택과 문장부호 등을 다룬다.


\subsection{한글과 한국어 문서}
\index{한국어}\index{Korean}\index{kotex}
\secbynomail{역자가 한국어판을 위해 추가}
현대적 텍 엔진인 \XeTeX 과 Lua\TeX 에서 한글을 식자하려면 \pai{fontspec} 패키지와 한글 글리프를 가진 글꼴만 
있으면 가능하다. \TeX\,Live에 포함된 한글 트루타입 폰트 ``은 글꼴''이 있으므로 
\begin{verbatim}
\fontspec{UnBatang.ttf} 한글 漢字
\end{verbatim}
와 같이 식자할 수 있다. 한자에 대해서도 마찬가지이다.

\pai{polyglossia} 패키지는 korean 언어 모듈을 가지고 있어서 한글을 지원한다.
글꼴을 잘 설정해주는 것에 주의하면 된다.
\begin{verbatim}
\usepackage{polyglossia}
\setotherlanguage{korean}
\newfontfamily\hangulfont{UnBatang.ttf}
\end{verbatim}
이제 korean 환경이나 \ci{textkorean} 명령을 사용할 수 있다. 원한다면 main language를 korean으로 설정하고 쓸 수도
있으며 이 경우에는 자동 생성 문자열, 행 간격, 일부 문장부호 등을 한글 문서에 알맞도록 설정해준다.

앞선 절에서 소개하는 \pai{xeCJK} 패키지로 한글을 식자하는 것도 어렵지 않다. 이 때 주의할 것은
중국어나 일본어에는 띄어쓰기가 없는 데 반해 우리 글은 띄어쓰기를 한다는 점이다. 그러므로 폰트 설정 시에 
이 사실을 명시해야 할 것이다.

\begin{verbatim}
  \usepackage{xeCJK}
  \xeCJKsetup{%
    CJKspace=true,%
    CJKecglue={}%
  }
  \setCJKmainfont{NanumMyeongjo}%
    [Ligatures=TeX,BoldFont={* ExtraBold},AutoFakeSlant]
\end{verbatim}

이상의 방법들은 한국어를 주요 언어로 하는 문서보다는 보조 언어로서 한글 표기가 필요할 경우에 쓸 수 있다.

\subsubsection[kotex]{\texorpdfstring{\koTeX}{koTeX}}
한국어를 주요 언어로 사용하고 한글로 문서를 작성한다면 \pai{kotex} 패키지군을 이용하는 것이 가장 바람직하다.
이 패키지군의 장점을 열거하면 다음과 같다.
\begin{enumerate} \firmlist
  \item \TeX\,Live에 포함되어 있어 별도의 설치가 필요없다.
  \item 다양한 텍 엔진에 대응한다. (pdf\LaTeX, \XeLaTeX, Lua\LaTeX, \hologo{plainTeX} 등)
  \item 자동조사(\pageref{autojosa}페이지를 보라)를 비롯하여 한국어를 올바로 식자하기 위한 기능들을 갖추고 있다.
  \item KTS(한국텍학회)가 공식적으로 지원하는 패키지로서 KTUG(\url{http://www.ktug.org})을 통하여 사용상의 도움을 얻을 수 있다.
  \item 단순한 문서에서 복잡한 문서나 단행본까지 \pai{kotex}으로 작성된 많은 예들이 있다.
  \item 확장된 기능(옛한글, CJK 언어지원, index 생성 유틸리티 등)과 클래스를 제공한다.
\end{enumerate}

\begin{lscommand}
  \ci{usepackage}\texttt{\{kotex\}}
\end{lscommand}

\pai{kotex} 패키지는 현재 엔진에 따라 동작이 조금씩 달라진다. 여기서는 \XeLaTeX\ 엔진을 중심으로 기술한다.
\XeLaTeX 이 이른바 `레거시 텍'과 가장 크게 다른 점은 시스템에 설치된 트루타입과 오픈타입 폰트를 \LaTeX{} 문서에서 사용할 수 있다는 점이다. \koTeX 도 한글 폰트를 자유롭게 사용할 수 있다. \XeLaTeX 에서 한글 문서의 폰트 문제에 대하여 \pageref{kofonts}페이지의 \ref{kofonts}절에 약간의 정보를 추가해두었다. 별다른 설정이 없으면 한글 본문글꼴로 `나눔명조'가 사용된다.

\begin{lscommand}
  \cs{usepackage}\verb|[hangul]{kotex}|
\end{lscommand}
\noindent \pai{kotex}에 \verb|[hangul]| 옵션을 부여하면 
단순히 한글을 식자하는 것을 넘어서 한글 문서에 필요한 요소들을 좀더 풍부하게 갖추어준다.
예를 들면 행 간격이 보기 좋을 정도로 늘어나고 목차의 제목이 ``Contents''가 아니라 ``차례''로 
바뀌며 \texttt{book}이나 \texttt{report}의 \cs{chapter}가 ``제\,1\,장''과 같은 형식으로 식자된다.

\begin{lscommand}
  \cs{documentclass}\verb|{oblivoir}|
\end{lscommand}
\pai{kotex} 패키지군 중에 한글 문서를 위한 클래스가 제공된다. \pai{oblivoir} 클래스를 사용하면 
\verb|\usepackage{kotex}| 없이 바로 한글 문서를 작성할 수 있고 한글 문서에 적합한 설정을
기본적으로 갖추어준다. 이 패키지는 \pai{memoir}를 
바탕으로 작성되었으므로 \pai{memoir}의 다양한 기능을 문서에서 사용할 수 있다.


\subsubsection{레거시 텍의 한글 패키지}

유니코드 텍 엔진이 보편화되기 전에는 한글을 표현하는 것이 쉽지 않은 일이었다. 
현재 레거시 텍을 위한 한글 패키지 가운데 중요한 것은 다음 세 가지이다.
\begin{description}
  \item[cjk-ko] \pai{kotex} 패키지군의 일부이지만 \pai{CJK} 패키지를 활용하도록 되어 있는 간단한 패키지이다.
  \item[kotex-utf] \pai{kotex} 패키지군의 일부로서 이전 \HLaTeX 을 계승하고 있는 것이다. 
  \item[CJK] 중국어, 일본어, 한국어 외에도 몇 가지 아시아 언어를 더 표현할 수 있는 패키지이다.
\end{description}
한글을 식자하는 폰트는 \pai{cjk-ko}와 \pai{kotex-utf}가 nanumtype1 또는 unfonts-type1을 활용하며
\pai{CJK}는 uhc 폰트를 쓴다. 이 이외에 쓸 수 있는 폰트는 거의 없다. 

현재도 pdf\LaTeX 을 활용해야 할 경우가 적지 않다. 그럴 적에 이 패키지들을 활용할 수 있다. 만약 \ci{usepackage}\verb|{kotex}|
문장이 있다면 기본적으로 \pai{kotex-utf}가 실행되게 되어 있다. 레거시 텍의 \pai{kotex}을 쓸 적에는 \pai{inputenc}와
함께 쓰지 않도록 주의하라.


% \section{The Space Between Words}
\section{단어 사이의 공백}

% To get a straight right margin in the output, \LaTeX{} inserts varying
% amounts of space between the words. It inserts slightly more space at
% the end of a sentence, as this makes the text more readable.  \LaTeX{}
% assumes that sentences end with periods, question marks or exclamation
% marks. If a period follows an uppercase letter, this is not taken as a
% sentence ending, since periods after uppercase letters normally occur in
% abbreviations.
조판 결과 양끝맞춤이 제대로 되도록 하려고 \LaTeX 은 단어 사이에 가변적인 공백을 넣는다.
문장의 끝에는 텍스트의 가독성을 높이기 위해 약간 더 많은 공백을 추가한다.
\LaTeX 은 마침표, 물음표, 느낌표가 오면 문장이 끝난 것으로 간주한다. 대문자 다음에 오는 
마침표는 보통 약어를 나타내기 위한 것으로 문장의 종지로 취급하지 않는다.

% Any exception from these assumptions has to be specified by the
% author. A backslash in front of a space generates a space that will
% not be enlarged. A tilde~`\verb|~|' character generates a space that cannot be
% enlarged and additionally prohibits a line break. The command
% \verb|\@| in front of a period specifies that this period terminates a
% sentence even when it follows an uppercase letter.
% \cih{"@} \index{~@ \verb.~.} \index{tilde@tilde ( \verb.~.)}
% \index{., space after}
이 가정에 예외가 되는 것들은 문서작성자가 그 사실을 알려주어야 한다. 스페이스 앞에 
백슬래시를 두면 이 공백의 폭을 변하지 않게 하는 것이다. 틸데 문자 `\verb|~|'는 
폭을 고정시키면서 줄나눔도 일어나지 않도록 하는 공백을 만든다.
마침표 앞의 \verb|\@| 명령은 이 마침표가 대문자 뒤에 나온 경우라도 문장의 끝임을 표시한다.
\cih{"@} \index{~@ \verb.~.} \index{tilde@tilde ( \verb.~.)}
\index{., space after}

% \begin{example}
% Mr.~Smith was happy to see her\\
% cf.~Fig.~5\\
% I like BASIC\@. What about you?
% \end{example}
\begin{example}
Mr.~Smith was happy to see her\\
cf.~Fig.~5\\
I like BASIC\@. What about you?
\end{example}

% The additional space after periods can be disabled with the command
% \begin{lscommand}
% \ci{frenchspacing}
% \end{lscommand}
% \noindent which tells \LaTeX{} \emph{not} to insert more space after a
% period than after an ordinary character. This is very common in
% non-English languages, except bibliographies. If you use
% \ci{frenchspacing}, the command \verb|\@| is not necessary.
문장의 종지 뒤에 추가 공백이 붙지 않도록 하려면
\begin{lscommand}
\ci{frenchspacing}
\end{lscommand}
\noindent 이라고 선언한다. \LaTeX 으로 하여금 마침표 뒤에 추가 공백을 삽입하지 말고
일반 문자처럼 취급하라고 지시하는 것이다.
비영어권 언어에서 (문헌목록은 예외) 일반적이다. \ci{frenchspacing}을 사용한다면 
\verb|\@| 명령은 불필요하다.

% \section{Titles, Chapters, and Sections}
\section{표제와 장절}

% To help the reader find his or her way through your work, you should
% divide it into chapters, sections, and subsections.  \LaTeX{} supports
% this with special commands that take the section title as their
% argument.  It is up to you to use them in the correct order.
독자의 독서를 용이하게 하기 위해 문서를 장, 절, 소절로 구분한다.
\LaTeX 에서는 절 표제를 인자로 취하는 특별한 명령으로 이를 표현한다.
올바르게 사용하는 것은 전적으로 문서작성자의 몫이다.

% The following sectioning commands are available for the
% \texttt{article} class: \nopagebreak
다음 장절 명령은 \texttt{article} 클래스에서 사용할 수 있다.

% \begin{lscommand}
% \ci{section}\verb|{...}|\\
% \ci{subsection}\verb|{...}|\\
% \ci{subsubsection}\verb|{...}|\\
% \ci{paragraph}\verb|{...}|\\
% \ci{subparagraph}\verb|{...}|
% \end{lscommand}
\begin{lscommand}
\ci{section}\verb|{...}|\\
\ci{subsection}\verb|{...}|\\
\ci{subsubsection}\verb|{...}|\\
\ci{paragraph}\verb|{...}|\\
\ci{subparagraph}\verb|{...}|
\end{lscommand}

% If you want to split your document into parts without influencing the
% section or chapter numbering use
% \begin{lscommand}
% \ci{part}\verb|{...}|
% \end{lscommand}
장과 절의 번호매김을 바꾸지 않으면서 문서를 1부, 2부 등으로 구분하려면 \ci{part} 명령을 쓴다.
\begin{lscommand}
\ci{part}\verb|{...}|
\end{lscommand}

% When you work with the \texttt{report} or \texttt{book} class,
% an additional top-level sectioning command becomes available
% \begin{lscommand}
% \ci{chapter}\verb|{...}|
% \end{lscommand}
\texttt{report}나 \texttt{book} 클래스로 작업하고 있다면 추가적인 상위 장절명령이 하나 더 있다.
\begin{lscommand}
\ci{chapter}\verb|{...}|
\end{lscommand}

% As the \texttt{article} class does not know about chapters, it is quite easy
% to add articles as chapters to a book.
% The spacing between sections, the numbering and the font size of the
% titles will be set automatically by \LaTeX.
\texttt{article} 클래스는 chapter가 없기 때문에 article을 모아서 chapter로 묶어 책으로 만들기 쉽다.
장절 표제의 간격, 번호매김, 폰트 크기 등은 클래스의 정의에 따라 \LaTeX 이 자동으로 설정한다.

% Two of the sectioning commands are a bit special:
% \begin{itemize}
% \item The \ci{part} command does
%   not influence the numbering sequence of chapters.
% \item The \ci{appendix} command does not take an argument. It just
%   changes the chapter numbering to letters.\footnote{For the article
%     style it changes the section numbering.}
% \end{itemize}
조금 특별한 장절명령이 두 가지 있다.
\begin{itemize} \firmlist
\item \ci{part} 
    명령은 장 번호에 영향을 주지 않는다.
\item \ci{appendix} 
    명령은 인자 없이 쓰인다. 장 번호 모양을 숫자가 아닌 문자로 바꾼다.\footnote{article 스타일에서는 장 번호가 아니라 절 번호가 바뀐다.}
\end{itemize}

% \LaTeX{} creates a table of contents by taking the section headings
% and page numbers from the last compile cycle of the document. The command
% \begin{lscommand}
% \ci{tableofcontents}
% \end{lscommand}
% \noindent expands to a table of contents at the place it
% is issued. A new
% document has to be compiled (``\LaTeX ed'') twice to get a
% correct \wi{table of contents}. Sometimes it might be
% necessary to compile the document a third time. \LaTeX{} will tell you
% when this is necessary.
\LaTeX 은 컴파일의 직전 단계에서 기억한 각 장절의 표제와 페이지 번호를 모아서 목차를 만든다.
\begin{lscommand}
\ci{tableofcontents}
\end{lscommand}
\noindent 이 명령이 주어진 위치에 목차를 넣는다.
새 문서라면 두 번 컴파일(``\LaTeX\ 실행'')해야 올바른 \wi{목차}[table of contents]를 생성한다.
세 번 컴파일해야 하는 경우도 있는데 추가 컴파일이 필요한지 여부를 \LaTeX 이 알려준다.

% All sectioning commands listed above also exist as ``starred''
% versions.  A ``starred'' version of a command is built by adding a
% star \verb|*| after the command name.  This generates section headings
% that do not show up in the table of contents and are not
% numbered. The command \verb|\section{Help}|, for example, would become
% \verb|\section*{Help}|.
위에 열거한 장절 명령에는 ``별표 붙은'' 명령도 있다. ``별표 붙은'' 명령이란 명령 이름 뒤에 
\verb|*|를 붙여서 지시하는 것을 말한다. 이 경우에는 절 제목이 목차에 나타나지 않으며 절 번호도 붙지 않는다.
예를 들면 \verb|\section{Help}| 대신 \verb|\section*{Help}|로 하는 것이다.

% Normally the section headings show up in the table of contents exactly
% as they are entered in the text. Sometimes this is not possible,
% because the heading is too long to fit into the table of contents. The
% entry for the table of contents can then be specified as an
% optional argument in front of the actual heading.
대개 장절의 표제는 입력된 그대로 목차에 나타난다. 그런데 그 표제가 너무 길어서 
목차에 넣기에는 적절치 않을 때가 있다. 목차에 넣을 (짧은) 표제를 별도로 지정하려면 
실제 표제 앞에 옵션 인자로 주면 된다.

% \begin{code}
% \verb|\chapter[Title for the table of contents]{A long|\\
% \verb|    and especially boring title, shown in the text}|
% \end{code}
\begin{code}
\verb|\chapter[Title for the table of contents]{A long|\\
\verb|    and especially boring title, shown in the text}|
\end{code}

% The \wi{title} of the whole document is generated by issuing a
% \begin{lscommand}
% \ci{maketitle}
% \end{lscommand}
% \noindent command. The contents of the title have to be defined by the commands
% \begin{lscommand}
% \ci{title}\verb|{...}|, \ci{author}\verb|{...}|
% and optionally \ci{date}\verb|{...}|
% \end{lscommand}
% \noindent before calling \verb|\maketitle|. In the argument to \ci{author}, you can supply several names separated by \ci{and} commands.
문서 전체의 \wi{타이틀}[title]을 만드는 명령은 다음과 같다.
\begin{lscommand}
\ci{maketitle}
\end{lscommand}
\noindent 타이틀을 만들기 위해 필요한 내용을 다음 명령으로 정의한다.
\begin{lscommand}
\ci{title}\verb|{...}|, \ci{author}\verb|{...}|,
\ci{date}\verb|{...}| (생략가능)
\end{lscommand}
\noindent \verb|\maketitle|을 부르기 전에 이 명령이 실행되어 있어야 한다.
\ci{author} 명령으로 여러 사람을 열거하려 할 때는 \ci{and} 명령으로 각각을 구분한다.


% An example of some of the commands mentioned above can be found in
% Figure~\ref{document} on page~\pageref{document}.
\pageref{document}페이지의 그림 \ref{document}에 위에 언급한 명령이 사용된 사례를 볼 수 있다.

% Apart from the sectioning commands explained above, \LaTeXe{}
% introduced three additional commands for use with the \verb|book| class.
% They are useful for dividing your publication. The commands alter
% chapter headings and page numbering to work as you would expect in
% a book:
위에 설명한 장절 명령과는 별도로 \LaTeXe 에서 \verb|book| 클래스에서 쓰는 세 가지 명령이 추가되었다.
출판물을 구획하는 데 유용한 명령들이다. 실제 출판된 단행본에서 볼 수 있는 것처럼 장의 표제나 페이지 번호 등을 변경한다.
% \begin{description}
% \item[\ci{frontmatter}] should be the very first command after
%   the start of the document body (\verb|\begin{document}|). It will switch page numbering to Roman
%     numerals and sections will be non-enumerated as if you were using
%     the starred sectioning commands (eg \verb|\chapter*{Preface}|)
%     but the sections will still show up in the table of contents.
% \item[\ci{mainmatter}] comes right before the first chapter of
%   the book. It turns on Arabic page numbering and restarts the page
%   counter.
% \item[\ci{appendix}] marks the start of additional material in your
%   book. After this command chapters will be numbered with letters.
% \item[\ci{backmatter}] should be inserted before the very last items
%   in your book, such as the bibliography and the index. In the standard
%   document classes, this has no visual effect.
% \end{description}
\begin{description}
  \item[\ci{frontmatter}] 이 명령은 문서 본문의 시작(\verb|\begin{document}|) 직후에 제일 먼저 와야 한다. 페이지 번호 모양을 로마 숫자로 바꾸고 장절 표제에는 별표 붙은 명령을 준 것처럼 번호가 붙지 않는다. 그러나 목차에는 표제가 나타날 것이다.
  \item[\ci{mainmatter}] 이 명령은 책 본문의 첫 장 직전에 온다. 페이지 번호 모양이 아라비아 숫자로 매겨지고 페이지 번호를 1부터 새로 시작한다.
  \item[\ci{appendix}] 책의 부록 부분의 시작임을 표시한다. 이 명령 이후 chapter들은 번호가 알파벳 문자로 붙는다.
  \item[\ci{backmatter}] 책의 마지막 부분 바로 앞에 두는 명령이다. 이 뒤로는 참고문헌 목록, 색인 등이 온다. 표준 클래스에서 특별한 모양의 변화는 없다.
\end{description} 


% \section{Cross References}
\section{교차참조}

% In books, reports and articles, there are often
% \wi{cross-references} to figures, tables and special segments of text.
% \LaTeX{} provides the following commands for cross referencing
% \begin{lscommand}
% \ci{label}\verb|{|\emph{marker}\verb|}|, \ci{ref}\verb|{|\emph{marker}\verb|}|
% and \ci{pageref}\verb|{|\emph{marker}\verb|}|
% \end{lscommand}
% \noindent where \emph{marker} is an identifier chosen by the user. \LaTeX{}
% replaces \verb|\ref| by the number of the section, subsection, figure,
% table, or theorem after which the corresponding \verb|\label| command
% was issued. \verb|\pageref| prints the page number of the
% page where the \verb|\label| command occurred.\footnote{Note that these commands
%   are not aware of what they refer to. \ci{label} just saves the last
%   automatically generated number.} As with section titles and page numbers for the table of contents,
% the numbers from the previous compile cycle are used.
단행본, 보고서, 논문 등에는 그림, 표, 텍스트의 특정 부분에 대하여 \wi{교차참조}[cross-references]하는 경우가 많다.
\LaTeX 은 교차참조를 위하여 다음과 같은 명령을 마련하고 있다.
\begin{lscommand}
\ci{label}\verb|{|\emph{marker}\verb|}|, \ci{ref}\verb|{|\emph{marker}\verb|}|, \ci{pageref}\verb|{|\emph{marker}\verb|}|
\end{lscommand}
\noindent 여기서 \emph{marker}는 문서작성자가 지시하는 식별자이다. \verb|\ref| 명령을 만나면 \LaTeX 은 
대응하는 \verb|\label| 명령이 위치한 곳에 있는 절, 소절, 그림, 표, 정리(theorem) 들의 번호를 취하여 그 자리에 식자한다.
\verb|\pageref|은 그 \verb|\label|이 있는 페이지의 번호를 인쇄한다.\footnote{%
  이 명령이 가리키는 바가 무엇인지를 명령 자체가 알고 있는 것이 아니라는 점을 기억하자. 
  \ci{label}은 단지 마지막에 얻은 번호를 저장할 뿐이다.
}
표제 텍스트와 페이지 번호로 목차를 만들 때와 마찬가지로 이 숫자(번호)들은 직전 컴파일 때에 저장해둔 것을 사용하므로 
교차참조 숫자가 잘 나타나려면 두 번 이상의 컴파일이 필요하다.\trfnote{%
  ``방정식 (1)''과 같은 모양을 만드는 특별한 참조 명령 \cs{eqref}이 있다.
  이것은 수학식의 번호를 참조하는데 \pai{amsmath} 패키지에서 제공하는 것이다. 
  다음 장의 수학식 조판에서 사용하고 있다.
}

% \begin{example}
% A reference to this subsection
% \label{sec:this} looks like:
% ``see section~\ref{sec:this} on
% page~\pageref{sec:this}.''
% \end{example}
\begin{example}
A reference to this subsection
\label{sec:this} looks like:
``see section~\ref{sec:this} on
page~\pageref{sec:this}.''
\end{example}

\section{조사의 선택}\label{autojosa}
\secbynomail{한국어판을 위하여 역자가 추가}
한국어의 조사 중에는 앞말의 끝소리에 따라 형태가 교체되는 것이 있다.
한국어 문서를 작성할 때 교차참조 기능을 사용하게 되면 문서작성자가 ``\texttt{\cs{ref}\{<label>\}을}’’이라고 
적는 시점에서 \verb|\ref|가 만들어내는 숫자가 얼마일지 알 수 없다. 그러므로
그 뒤에 올 조사가 `을'이 될지 `를'이 될지를 나중에 만들어진 숫자를 보고 수정해야 한다.
이런 불편을 없애고 조사의 형태를 \LaTeX 이 직접 선택할 수 있게 해주는 기능이 \koTeX 에 있다.
만약 \koTeX 을 사용하고 있다면 위의 사례는 다음과 같이 백슬래시를 붙여서 입력한다.
\begin{example}
  \ref{autojosa}\을 보아라
\end{example}
\noindent 이를 ``\emph{자동조사}''라고 부른다. 자동조사 명령은 다음과 같다.\index{자동조사}\index{kotex}
\begin{lscommand}
  \cs{은} \cs{는}
  \cs{이} \cs{가}
  \cs{을} \cs{를}
  \cs{와} \cs{과}
  \cs{로} \cs{으로}
  \cs{라} \cs{이라}
\end{lscommand}
\noindent 짝을 이루는 조사 명령은 어떤 것을 써도 결과가 같다.

자동조사는 교차참조의 경우뿐 아니라 숫자 방식의 인용(\cs{cite})에 대해서도 믿을 만하게 동작한다.

% \section{Footnotes}
\section{각주}
% With the command
% \begin{lscommand}
% \ci{footnote}\verb|{|\emph{footnote text}\verb|}|
% \end{lscommand}
% \noindent a footnote is printed at the foot of the current page.  Footnotes
% should always be put\footnote{``put'' is one of the most common
%   English words.} after the word or sentence they refer to. Footnotes
% referring to a sentence or part of it should therefore be put after
% the comma or period.\footnote{Note that footnotes
%   distract the reader from the main body of your document. After all,
%   everybody reads the footnotes---we are a curious species, so why not
%   just integrate everything you want to say into the body of the
%   document?\footnotemark}
% \footnotetext{A guidepost doesn't necessarily go where it's pointing to :-).}
\wi{각주}를 달려면 다음 명령을 쓴다.
\begin{lscommand}
\ci{footnote}\verb|{|\emph{footnote text}\verb|}|
\end{lscommand}

\noindent 그러면 현재 페이지의 하단부에 각주가 인쇄된다.
각주는 문장 전체나 그 일부에 대해 붙는\footnote{각주를 붙인 예} 것이므로 쉼표나 마침표 뒤에 붙어야 한다.\footnote{%
  각주는 문서의 본문을 읽는 독자의 주의를 다른 곳으로 돌리게 한다는 점을 기억하라.
  아무튼 각주까지 모두 읽는 우리 같은 사람은 좀 별난 축에 든다. 그러니 말하고 싶은 것이 있으면 모두 본문에 포함시키는 것이 좋지 않을까?\footnotemark}%
\footnotetext{각주에 다시 각주를 붙여본 예. \cs{footnote}를 각주 내에 다시 쓸 수 없기 때문에 여기서는 \cs{footnotemark}와 \cs{footnotetext}를 
따로 이용하였다.}%
\textsuperscript{,}\trfnote{%
  `미주'라 하여 페이지 바닥에 주를 달지 않고 장이나 책의 마지막에 모아두는 방법이 있다.
  이를 구현하려면 \pai{endnotes} 또는 \pai{enotez} 패키지를 이용할 수 있다.
}

% \begin{example}
% Footnotes\footnote{This is
%   a footnote.} are often used
% by people using \LaTeX.
% \end{example}
\begin{example}
Footnotes\footnote{This is
  a footnote.} are often used
by people using \LaTeX.
\end{example}


% \section{Emphasized Words}
\section{단어의 강조}

% If a text is typed using a typewriter, important words are
%   \texttt{emphasized by \underline{underlining} them.}
% \begin{lscommand}
% \ci{underline}\verb|{|\emph{text}\verb|}|
% \end{lscommand}
% In printed books,
% however, words are emphasized by typesetting them in an \emph{italic}
% font.  As an author you shouldn't care either way. The important bit is, to tell \LaTeX{} that a
% particular bit of text is important and should be emphasized. Hence the command
% \begin{lscommand}
% \ci{emph}\verb|{|\emph{text}\verb|}|
% \end{lscommand}
% \noindent to emphasize text. What the command actually does with
% its argument depends on the context:
타자기를 쓰던 시절에는 중요한 단어에 \texttt{\underline{밑줄을 그어서} 강조했다}.
\begin{lscommand}
\ci{underline}\verb|{|\emph{text}\verb|}|
\end{lscommand}
출판물에서는 타자친 원고의 밑줄이 그어진 부분을 \emph{italic}으로 식자하여 \wi{강조}임을 표시했다.
문서작성자는 밑줄이든 이탤릭이든 신경쓸 필요가 없다. 중요한 점은 \LaTeX 에게 이 부분이 특히 중요하여 
강조로 표현되어야 한다는 것을 알려주는 것이다.
\begin{lscommand}
\ci{emph}\verb|{|\emph{text}\verb|}|
\end{lscommand}
\noindent 이 명령으로 강조 어구임을 표시할 수 있다. 실제로 인자가 어떻게 인쇄되는가는 
맥락에 따라 달라진다. 다음 보기를 보라.\trfnote{%
  한글 폰트의 폰트가족에는 이탤릭체가 없다. 따라서 강조에 이탤릭을 쓰지 않는다.
  종래 ``우사체(右斜體)'' 또는 ``기울임꼴''이라는 것으로 강제로 오른쪽으로 기울어지게 변형(fake-slanted)한 글자를
  이탤릭체 대용으로 쓰던 관행이 없지는 않지만 오늘날 적절한 방법으로서 추천하기는 어렵다.
  한글로 된 문장에서 일부 단어를 강조하기 위해서 \dotemph{글자 위에 점을 찍}거나
  \emph{서체를 바꾸는} 방법이 바람직하다.
}

% \begin{example}
% \emph{If you use
%   emphasizing inside a piece
%   of emphasized text, then
%   \LaTeX{} uses the
%   \emph{normal} font for
%   emphasizing.}
% \end{example}
\begin{example}
\emph{If you use
  emphasizing inside a piece
  of emphasized text, then
  \LaTeX{} uses the
  \emph{normal} font for
  emphasizing.}
\end{example}

% If you want control over font and font size, section \ref{sec:fontsize} on
% page \pageref{sec:fontsize} might provide some inspiration.
폰트의 종류나 크기를 바꾸고 싶다면 \pageref{sec:fontsize}페이지의 \ref{sec:fontsize}절이
약간 도움이 될 것이다.

% \section{Environments} \label{env}
\section{환경} \label{env}

% % To typeset special purpose text, \LaTeX{} defines many different
% % \wi{environment}s for all sorts of formatting:
% \begin{lscommand}
% \ci{begin}\verb|{|\emph{environment}\verb|}|\quad
%    \emph{text}\quad
% \ci{end}\verb|{|\emph{environment}\verb|}|
% \end{lscommand}
% \noindent Where \emph{environment} is the name of the environment. Environments can be
% nested within each other as long as the correct nesting order is
% maintained.
% \begin{code}
% \verb|\begin{aaa}...\begin{bbb}...\end{bbb}...\end{aaa}|
% \end{code}
\begin{lscommand}
\ci{begin}\verb|{|\emph{environment}\verb|}|\quad
   \emph{text}\quad
\ci{end}\verb|{|\emph{environment}\verb|}|
\end{lscommand}
\noindent 이 예제에서 \emph{environment}는 \wi{환경}[environment]의 이름이다.
하나의 환경 안에 또다시 환경이 올 수도 있다. 다만 열고 닫는 순서를 잘 지켜야 한다.
\begin{code}
\verb|\begin{aaa}...\begin{bbb}...\end{bbb}...\end{aaa}|
\end{code}

% \noindent In the following sections all important environments are explained.
\noindent 이하 소절에서 중요한 환경을 설명하겠다.

% \subsection{Itemize, Enumerate, and Description}
\subsection{리스트 문단(Itemize, Enumerate, Description)}

% The \ei{itemize} environment is suitable for simple lists, the
% \ei{enumerate} environment for enumerated lists, and the
% \ei{description} environment for descriptions.
% \cih{item}
\ei{itemize}는 단순 리스트,
\ei{enumerate}는 번호붙은 리스트,
\ei{description}은 설명형 리스트를 위한 환경이다. \cih{item}
각 항목은 \verb|\item|으로 지시해야 하며 하나 이상의 \verb|\item|이 반드시 있어야 한다.
\index{리스트 문단}

% \begin{example}
% \flushleft
% \begin{enumerate}
% \item You can nest the list
% environments to your taste:
% \begin{itemize}
% \item But it might start to
% look silly.
% \item[-] With a dash.
% \end{itemize}
% \item Therefore remember:
% \begin{description}
% \item[Stupid] things will not
% become smart because they are
% in a list.
% \item[Smart] things, though,
% can be presented beautifully
% in a list.
% \end{description}
% \end{enumerate}
% \end{example}
\begin{example}
\flushleft
\begin{enumerate}
\item You can nest the list
environments to your taste:
\begin{itemize}
\item But it might start to
look silly.
\item[-] With a dash.
\end{itemize}
\item Therefore remember:
\begin{description}
\item[Stupid] things will not
become smart because they are
in a list.
\item[Smart] things, though,
can be presented beautifully
in a list.
\end{description}
\end{enumerate}
\end{example}

% \subsection{Flushleft, Flushright, and Center}
\subsection{문단의 정렬(Flushleft, Flushright, Center)}

% The environments \ei{flushleft} and \ei{flushright} generate
% paragraphs that are either left- or \wi{right-aligned}. \index{left
%   aligned} The \ei{center} environment generates centred text. If you
% do not issue \ci{\bs} to specify line breaks, \LaTeX{} will
% automatically determine line breaks.
\ei{flushleft}와 \ei{flushright} 환경은 문단을 왼쪽 또는 오른쪽으로 
몰아서 정렬해준다. \ei{center} 환경은 텍스트를 가운데정렬한다.
\ci{\bs}로 줄나눔을 강제하지 않으면 \LaTeX 이 자동으로 줄을 나눈다.
\index{정렬!왼쪽정렬}\index{정렬!오른쪽정렬}\index{정렬!가운데정렬}

% \begin{example}
% \begin{flushleft}
% This text is\\ left-aligned.
% \LaTeX{} is not trying to make
% each line the same length.
% \end{flushleft}
% \end{example}
\begin{example}
\begin{flushleft}
This text is\\ left-aligned.
\LaTeX{} is not trying to make
each line the same length.
\end{flushleft}
\end{example}

\vspace{-.5\onelineskip}

% \begin{example}
% \begin{flushright}
% This text is right-\\aligned.
% \LaTeX{} is not trying to make
% each line the same length.
% \end{flushright}
% \end{example}
\begin{example}
\begin{flushright}
This text is right-\\aligned.
\LaTeX{} is not trying to make
each line the same length.
\end{flushright}
\end{example}

\vspace{-.5\onelineskip}

% \begin{example}
% \begin{center}
% At the centre\\of the earth
% \end{center}
% \end{example}
\begin{example}
\begin{center}
At the centre\\of the earth
\end{center}
\end{example}

% \subsection{Quote, Quotation, and Verse}
\subsection{인용문과 운문(Quote, Quotation, Verse)}

% The \ei{quote} environment is useful for quotes, important phrases and
% examples.
\ei{quote}는 \wi{인용문}, 중요 구절, 예시 문장을 위해 사용할 수 있는 환경이다.

%다음 예제로 활용한 문장은 이런 뜻이다:
%
%``행의 길이에 대한 타이포그래피 규칙은 다음과 같다.
%\begin{quote}
%  평균적으로 한 줄에 들어가는 글자는 66자를 넘지 않아야 한다.
%\end{quote}
%이것이 \LaTeX 으로 조판한 문서의 상하좌우 여백이 커보이는 이유이며
%신문에서 다단조판을 활용하는 이유이기도 하다.''

% \begin{example}
% A typographical rule of thumb
% for the line length is:
% \begin{quote}
% On average, no line should
% be longer than 66 characters.
% \end{quote}
% This is why \LaTeX{} pages have
% such large borders by default
% and also why multicolumn print
% is used in newspapers.
% \end{example}
\begin{example}
A typographical rule of thumb
for the line length is:
\begin{quote}
On average, no line should
be longer than 66 characters.
\end{quote}
This is why \LaTeX{} pages have
such large borders by default
and also why multicolumn print
is used in newspapers.
\end{example}



% There are two similar environments: the \ei{quotation} and the
% \ei{verse} environments. The \texttt{quotation} environment is useful
% for longer quotes going over several paragraphs, because it indents the
% first line of each paragraph. The \texttt{verse} environment is useful for poems
% where the line breaks are important. The lines are separated by
% issuing a \ci{\bs} at the end of a line and an empty line after each
% verse.
이와 유사한 환경이 두 가지 더 있다. 하나는 \ei{quotation}이고 다른 하나는 \ei{verse}이다.
\texttt{quotation} 환경은 인용문이 몇 단락 이상을 포함할 정도로 길 때 사용할 수 있다. 인용문의
각 단락은 들여쓰기를 한다.

\texttt{verse} 환경은 시와 같이 줄나눔이 중요한 \wi{운문}을 조판하는 데 유용하다.
행(行)의 끝은 \ci{\bs}로 표시하며, 연(聯)과 연 사이에 빈 줄을 둔다.


% \begin{example}
% I know only one English poem by
% heart. It is about Humpty Dumpty.
% \begin{flushleft}
% \begin{verse}
% Humpty Dumpty sat on a wall:\\
% Humpty Dumpty had a great fall.\\
% All the King's horses and all
% the King's men\\
% Couldn't put Humpty together
% again.
% \end{verse}
% \end{flushleft}
% \end{example}
\begin{example}
I know only one English poem by
heart. It is about Humpty Dumpty.
\begin{flushleft}
\begin{verse}
Humpty Dumpty sat on a wall:\\
Humpty Dumpty had a great fall.\\
All the King's horses and all
the King's men\\
Couldn't put Humpty together
again.
\end{verse}
\end{flushleft}
\end{example}

% \subsection{Abstract}
\subsection{요약문(Abstract)}

% In scientific publications it is customary to start with an abstract which
% gives the reader a quick overview of what to expect. \LaTeX{} provides the
% \ei{abstract} environment for this purpose. Normally \ei{abstract} is used
% in documents typeset with the article document class.
과학문헌은 독자에게 내용에 대한 개관을 제공하는 \wi{요약문}을 앞에 두는 것이 관례이다.
\LaTeX 에서는 \ei{abstract} 환경을 사용하여 요약문을 만들 수 있다. 일반적으로 \ei{abstract}는 
article 클래스를 사용하는 문서에서 쓰인다.

% \newenvironment{abstract}%
%         {\begin{center}\begin{small}\begin{minipage}{0.8\textwidth}}%
%         {\end{minipage}\end{small}\end{center}}
% \begin{example}
% \begin{abstract}
% The abstract abstract.
% \end{abstract}
% \end{example}
\provideenvironment{abstract}%
        {\begin{center}\begin{small}\begin{minipage}{0.8\textwidth}}%
        {\end{minipage}\end{small}\end{center}}
\begin{example}
\begin{abstract}
The abstract abstract.
\end{abstract}
\end{example}

% \subsection{Printing Verbatim}
\subsection{그대로 보이기(Verbatim)}

% Text that is enclosed between \verb|\begin{|\ei{verbatim}\verb|}| and
% \verb|\end{verbatim}| will be directly printed, as if typed on a
% typewriter, with all line breaks and spaces, without any \LaTeX{}
% command being executed.
\verb|\begin{|\ei{verbatim}\verb|}|와 \verb|\end{verbatim}|으로 둘러싸인 부분은
마치 타자기로 친 것과 같이 줄나눔, 공백 등이 입력된 그대로 나타나며 그 안에 있는 \LaTeX{}
명령이 전혀 실행되지 않는다.

% Within a paragraph, similar behavior can be accessed with
% \begin{lscommand}
% \ci{verb}\verb|+|\emph{text}\verb|+|
% \end{lscommand}
% \noindent The \verb|+| is just an example of a delimiter character. Use any
% character except letters, \verb|*| or space. Many \LaTeX{} examples in this
% booklet are typeset with this command.
한 문단 안에서 다음과 같이 하면 비슷한 결과를 얻을 수 있다.
\begin{lscommand}
\ci{verb}\verb|+|\emph{text}\verb|+|
\end{lscommand}
\noindent \verb|+|는 시작과 끝을 나타내는 경계문자의 한 예시이다. 문자(letter), \verb|*|, 공백(space)을
제외한 다른 글자를 사용할 수 있다. 이 문서의 많은 \LaTeX{} 코딩의 예들이 이 명령을 이용해서 작성되었다.

% \begin{example}
% The \verb|\ldots| command \ldots

% \begin{verbatim}
% 10 PRINT "HELLO WORLD ";
% 20 GOTO 10
% \end{verbatim}
% \end{example}

% \begin{example}
% \begin{verbatim*}
% the starred version of
% the      verbatim
% environment emphasizes
% the spaces   in the text
% \end{verbatim*}
% \end{example}
\begin{example}
The \verb|\ldots| command \ldots

\begin{verbatim}
10 PRINT "HELLO WORLD ";
20 GOTO 10
\end{verbatim}
\end{example}

\vspace{-.5\onelineskip}

% special case: 
% conflict between memoir's verbatim and minted package
\noindent 
\noindent\begin{adjustwidth}{0pt}{-\margheadwidth}
\noindent\hspace*{-1.5em}\begin{minipage}[c]{.49\linewidth}%
    \setstretch{1.0}%
    \noindent
\begin{tcblisting}{listing only, frame empty,minted options={fontsize=\small\ttfamily}}
\begin{verbatim*}
the starred version of
the      verbatim
environment emphasizes
the spaces   in the text
\end{verbatim*}
\end{tcblisting}
  \end{minipage}\fboxsep=8pt\hspace{.3em}%
    \setstretch{1.0}%
\begin{tcblisting}{listing only,colback=white,%frame empty,%
    sharp corners,width=.52\linewidth,boxrule=.66pt,valign=center,box align=center,
    minted options={fontsize=\normalsize\ttfamily,showspaces}}
the starred version of
the      verbatim
environment emphasizes
the spaces   in the text
\end{tcblisting}
\end{adjustwidth}


\medskip

% The \ci{verb} command can be used in a similar fashion with a star:
\ci{verb} 명령은 별표 붙은 명령의 형태로 사용할 수 있다. 어떤 점이 다른지는 다음 예를 보라.

% \begin{example}
% \verb*|like   this :-) |
% \end{example}
\begin{example}
\verb*|like   this :-) |
\end{example}

% The \texttt{verbatim} environment and the \verb|\verb| command may not be used
% within parameters of other commands.
\texttt{verbatim} 환경과 \verb|\verb| 명령은 다른 명령의 인자 안에서는 쓸 수 없다.

% \subsection{Tabular}
\subsection{표(Tabular)}

% \newcommand{\mfr}[1]{\framebox{\rule{0pt}{0.7em}\texttt{#1}}}
\newcommand{\mfr}[1]{\framebox{\rule{0pt}{0.7em}\texttt{#1}}}

% The \ei{tabular} environment can be used to typeset beautiful
% \wi{table}s with optional horizontal and vertical lines. \LaTeX{}
% determines the width of the columns automatically.
\ei{tabular} 환경은 \wi{표}[table]를 아름답게 조판하는 데 사용한다. 표에 가로선이나 세로선을 그을 수도 있다.
컬럼의 폭은 자동으로 결정된다.

% The \emph{table spec} argument of the
% \begin{lscommand}
% \verb|\begin{tabular}[|\emph{pos}\verb|]{|\emph{table spec}\verb|}|
% \end{lscommand}
% \noindent command defines the format of the table. Use an \mfr{l} for a column of
% left-aligned text, \mfr{r} for right-aligned text, and \mfr{c} for
% centred text; \mfr{p\{\emph{width}\}} for a column containing justified
% text with line breaks, and \mfr{|} for a vertical line.
\begin{lscommand}
\verb|\begin{tabular}[|\emph{pos}\verb|]{|\emph{table spec}\verb|}|
\end{lscommand}
여기서 \emph{table spec} 인자는 표의 형태를 결정하는 명령이다. \mfr{l}은 해당 열(column)의 텍스트가
왼쪽정렬 되도록 하라는 것이고 \mfr{r}은 오른쪽정렬 하라는 것이다. 가운데정렬은 \mfr{c}로 지시한다.
\mfr{p\{\emph{width}\}}는 컬럼의 내용을 \emph{width}에 맞추어 줄나눔하고 양끝정렬하라는 것을 의미한다.
\mfr{|}는 세로줄이다.

% If the text in a column is too wide for the page, \LaTeX{} won't
% automatically wrap it. Using \mfr{p\{\emph{width}\}} you can define
% a special type of column which will wrap-around the text as in a normal paragraph.
컬럼에 놓인 텍스트가 너무 길어서 페이지 폭에 넘치는 경우라도 \LaTeX 이 그것을 자동으로 자르지 않는다.
\mfr{p\{\emph{width}\}}를 지정하여 해당 컬럼에 오는 텍스트를 일반 문단처럼 정렬되도록 할 수 있다.

% The \emph{pos} argument specifies the vertical position of the table
% relative to the baseline of the surrounding text.  Use one of the
% letters \mfr{t}, \mfr{b} and \mfr{c} to specify table
% alignment at the top, bottom or centre.
\emph{pos} 인자는 인접 텍스트의 베이스라인에 대하여 표가 놓일 수직 위치를 지정한다.
\mfr{t}, \mfr{b}, \mfr{c} 가운데 하나를 쓸 수 있는데 각각 top, bottom, center를 의미한다.

% Within a \texttt{tabular} environment, \texttt{\&} jumps to the next
% column, \ci{\bs} starts a new line and \ci{hline} inserts a horizontal
% line.  Add partial lines by using \ci{cline}\texttt{\{}$i$\texttt{-}$j$\texttt{\}},
% where $i$ and $j$ are the column numbers the line should extend over.
\texttt{tabular} 환경 안에서 \texttt{\&}는 다음 컬럼으로 이동을 표시한다.
\ci{\bs}는 줄을 나누라는 뜻이며 \ci{hline}은 가로선을 긋게 하는 것이다.
일부 컬럼에만 걸치는 부분 가로선을 그으려면 \ci{cline}\texttt{\{}$i$\texttt{-}$j$\texttt{\}}와 같이 하는데
이 때 $i$와 $j$는 부분 가로선이 그어질 범위에 해당하는 컬럼의 번호이다.

% \index{"|@ \verb."|.}
\index{"|@ \verb."|.}

% \begin{example}
% \begin{tabular}{|r|l|}
% \hline
% 7C0 & hexadecimal \\
% 3700 & octal \\ \cline{2-2}
% 11111000000 & binary \\
% \hline \hline
% 1984 & decimal \\
% \hline
% \end{tabular}
% \end{example}

% \begin{example}
% \begin{tabular}{|p{4.7cm}|}
% \hline
% Welcome to Boxy's paragraph.
% We sincerely hope you'll
% all enjoy the show.\\
% \hline
% \end{tabular}
% \end{example}
\begin{example}
\begin{tabular}{|r|l|}
\hline
7C0 & hexadecimal \\
3700 & octal \\ \cline{2-2}
11111000000 & binary \\
\hline \hline
1984 & decimal \\
\hline
\end{tabular}
\end{example}

\vspace{-.5\onelineskip}

\begin{example}
\begin{tabular}{|p{4.7cm}|}
\hline
Welcome to Boxy's paragraph.
We sincerely hope you'll
all enjoy the show.\\
\hline
\end{tabular}
\end{example}

% The column separator can be specified with the \mfr{@\{...\}}
% construct. This command kills the inter-column space and replaces it
% with whatever is between the curly braces.  One common use for
% this command is explained below in the decimal alignment problem.
% Another possible application is to suppress leading space in a table with
% \mfr{@\{\}}.
\wi{컬럼구분자}는 \mfr{@\{...\}} 형식으로 지시할 수도 있다.
이 명령은 컬럼 사이의 공백을 없애고 그 부분을 중괄호 사이에 온 것으로 대체한다.
이 명령의 사용례가 아래 소수점 정렬 문제를 설명하는 데서 사용되고 있다.
그리고 표의 여분 공백을 제거하기 위하여 \mfr{@\{\}}를 사용하는 방식으로 응용가능하다.

% \begin{example}
% \begin{tabular}{@{} l @{}}
% \hline
% no leading space\\
% \hline
% \end{tabular}
% \end{example}

% \begin{example}
% \begin{tabular}{l}
% \hline
% leading space left and right\\
% \hline
% \end{tabular}
% \end{example}
\begin{example}
\begin{tabular}{@{} l @{}}
\hline
no leading space\\
\hline
\end{tabular}
\end{example}

\vspace{-.5\onelineskip}

\begin{example}
\begin{tabular}{l}
\hline
leading space left and right\\
\hline
\end{tabular}
\end{example}

% %
% % This part by Mike Ressler
% %

% \index{decimal alignment} Since there is no built-in way to align
% numeric columns to a decimal point,\footnote{If the `tools' bundle is
%   installed on your system, have a look at the \pai{dcolumn} package.}
% we can ``cheat'' and do it by using two columns: a right-aligned
% integer and a left-aligned fraction. The \verb|@{.}| command in the
% \verb|\begin{tabular}| line replaces the normal inter-column spacing with
% just a ``.'', giving the appearance of a single,
% decimal-point-justified column.  Don't forget to replace the decimal
% point in your numbers with a column separator (\verb|&|)! A column label
% can be placed above our numeric ``column'' by using the
% \ci{multicolumn} command.
\index{소수점 정렬}
아무런 패키지를 쓰지 않은 \LaTeX\ 자체만으로는 소수점을 기준으로 컬럼의 숫자를 정렬할 방법이 없다.\footnote{%
  `tools' 패키지묶음에 포함되어 있는 \pai{dcolumn} 패키지가 이런 일을 해준다.
}
``변칙적인'' 방법을 생각해보자. 두 개의 컬럼을 사용하되 왼쪽 컬럼에는 정수부분을, 오른쪽 컬럼에는 소수부분을 넣어두는 것이다.
그런 다음 \verb|@{.}|를 이용하여 컬럼 사이의 스페이스를 점(``.'')으로 대체하도록 \verb|\begin{tabular}|의 인자를
  지정하면 마치 두 컬럼이 소수점으로 정렬된 하나의 컬럼처럼 보이게 할 수 있다.
정수 부분 컬럼과 소수 부분 컬럼 사이에 컬럼 분리 문자(\verb|&|)가 있어야 한다는 점을 명심하자.
컬럼의 제목줄은 \ci{multicolumn} 명령을 이용하여 표시한다.

% \begin{example}
% \begin{tabular}{c r @{.} l}
% Pi expression       &
% \multicolumn{2}{c}{Value} \\
% \hline
% $\pi$               & 3&1416  \\
% $\pi^{\pi}$         & 36&46   \\
% $(\pi^{\pi})^{\pi}$ & 80662&7 \\
% \end{tabular}
% \end{example}

% \begin{example}
% \begin{tabular}{|c|c|}
% \hline
% \multicolumn{2}{|c|}{Ene} \\
% \hline
% Mene & Muh! \\
% \hline
% \end{tabular}
% \end{example}
\begin{example}
\begin{tabular}{c r @{.} l}
Pi expression       &
\multicolumn{2}{c}{Value} \\
\hline
$\pi$               & 3&1416  \\
$\pi^{\pi}$         & 36&46   \\
$(\pi^{\pi})^{\pi}$ & 80662&7 \\
\end{tabular}
\end{example}

\vspace{-.5\onelineskip}

\begin{example}
\begin{tabular}{|c|c|}
\hline
\multicolumn{2}{|c|}{Ene} \\
\hline
Mene & Muh! \\
\hline
\end{tabular}
\end{example}

% Material typeset with the tabular environment always stays together on one
% page. If you want to typeset long tables, you might want to use the
% \pai{longtable} environments.
\ei{tabular} 환경으로 식자되는 표는 항상 한 페이지 안에 놓인다. 표가 길어서 몇 페이지에 걸쳐
나누어져야 한다면 \pai{longtable} 패키지의 같은 이름의 환경을 사용할 수 있다.

% Sometimes the default \LaTeX{} tables do feel a bit cramped. So you may want
% to give them a bit more breathing space by setting a higher
% \ci{arraystretch} and \ci{tabcolsep} value.
가끔 \LaTeX 이 그린 표가 너무 협소하게 느껴진다면 \ci{arraystretch}와 \ci{tabcolsep} 값을
조정하여 넉넉하게 만들 수 있다.

% \begin{example}
% \begin{tabular}{|l|}
% \hline
% These lines\\\hline
% are tight\\\hline
% \end{tabular}

% {\renewcommand{\arraystretch}{1.5}
% \renewcommand{\tabcolsep}{0.2cm}
% \begin{tabular}{|l|}
% \hline
% less cramped\\\hline
% table layout\\\hline
% \end{tabular}}

% \end{example}
\begin{example}
\begin{tabular}{|l|}
\hline
These lines\\\hline
are tight\\\hline
\end{tabular}

{\renewcommand{\arraystretch}{1.5}
\renewcommand{\tabcolsep}{0.2cm}
\begin{tabular}{|l|}
\hline
less cramped\\\hline
table layout\\\hline
\end{tabular}}

\end{example}

% If you just want to grow the height of a single row in your table add an invisible vertical bar\footnote{In professional typesetting,
% this is called a \wi{strut}.}. Use a zero width \ci{rule} to implement this trick.
표의 특정 행(row)의 높이를 임의로 증가시키고 싶다면 보이지 않는 세로선\footnote{전문 조판에서 이것을 \wi{strut}라고 한다.}을 넣으면 된다.
폭이 0인 \ci{rule}을 그려서 이것을 구현해보자.\label{strutrule}

% \begin{example}
% \begin{tabular}{|c|}
% \hline
% \rule{1pt}{4ex}Pitprop \ldots\\
% \hline
% \rule{0pt}{4ex}Strut\\
% \hline
% \end{tabular}
% \end{example}
\begin{example}
\begin{tabular}{|c|}
\hline
\rule{1pt}{4ex}Pitprop \ldots\\
\hline
\rule{0pt}{4ex}Strut\\
\hline
\end{tabular}
\end{example}

% The \texttt{pt} and \texttt{ex} in the example above are \TeX{} units. Read more
% on units in table \ref{units} on page \pageref{units}.
여기에 쓰인 \texttt{pt}와 \texttt{ex}는 \TeX 의 길이 단위이다. \pageref{units}페이지의 표 \ref{units}를 보라.

% A number of extra commands, enhancing the tabular environment are available
% in the \pai{booktabs} package. It makes the creation of professional looking
% tables with proper spacing quite a bit simpler.
\ei{tabular} 환경을 확장하는 많은 추가적인 명령이 \pai{booktabs} 패키지에 정의되어 있다.
적절한 간격을 갖춘 전문적인 품위의 표를 상당히 쉽게 그리게 해준다.\trfnote{%
  \pai{booktabs} 패키지는 세로선이 없고 상하 가로선에 다른 두께를 주는 표를 그리는 데 유용하다.
  이런 유형의 표가 한글 문헌에서는 많지 않은 편이다.
}


% \section{Including Graphics and Images}\label{eps}
\section{그림 포함하기}\label{eps}

% As explained in the previous section \LaTeX{} provides the facilities to work with floating bodies,
% such as images or graphics, with the \texttt{figure} and
% \texttt{table} environments.

% A good set of commands for inclusion of graphics into these floating boddies is provided in the
% \pai{graphicx} package by D.~P.~Carlisle. It is part of a whole family
% of packages called the ``graphics''
% bundle.\footnote{\CTAN|pkg/graphics|}
떠다니는 개체에 그림을 넣는 명령은 D.~P.~Carlisle이 작성한 \pai{graphicx} 패키지에 의해 잘 마련되어 있다.
이 패키지는 ``graphics'' 패키지 묶음의 일부이다.\footnote{\CTAN|pkg/graphics|}
``떠다니는 개체''에 대해서는 \ref{floats}절에서 설명할 것이다.

% Use the following step by step guide to
% include a picture into your document:
다음 과정을 차례대로 따라해보기 바란다.

\enlargethispage*{2\onelineskip}

% \begin{enumerate}
\begin{enumerate} \firmlist 
% \item Export the picture from your graphics program in EPS, PDF, PNG or JPEG format.
\item 그래픽 프로그램에서 그림을 EPS, PDF, PNG, JPEG 중 하나의 포맷으로 내보내기(export)한다.
% \item If you exported your graphics as an EPS vector graphics, you have to convert it to PDF format
% prior to using it. There is a \texttt{epstopdf} command line tool that helps with this task.
% Note that it may be sensible to export EPS eventhough your software can export PDF too, as PDFs often are full page and will
% this get very small when imported into a document. EPS on the other hand come with a bounding box showing the extent of the actual graphics.
\item 그림을 EPS 벡터 그래픽으로 저장하였다면 미리 PDF 포맷으로 변환해둔다. 명령행 프로그램 \texttt{epstopdf}을 이용하여 변환할 수 있다. 
물론 PDF로 내보내기하여도 좋다. 다만 
EPS로 저장한 후에 변환하는 것이 더 나을 때가 있는데 소프트웨어가 PDF로 저장할 수 있지만 그 PDF가 페이지 전체 크기로 작성되어서 문서에 들여오면 실제 그림이 아주 작아져버리는 경우가 가끔 있기 때문이다. 이럴 때는 바운딩박스를 잘 갖추고 있는 EPS로 내보낸 다음 변환하는 것이 실제 크기의 그림을 얻을 수 있는 한 방법이 된다.
% \item Load the \textsf{graphicx} package in the preamble of the input
%   file with
% \begin{lscommand}
% \verb|\usepackage{graphicx}|
% \end{lscommand}
% \item Use the command
% \begin{lscommand}
% \ci{includegraphics}\verb|[|\emph{key}=\emph{value}, \ldots\verb|]{|\emph{file-name}\verb|}|
% \end{lscommand}
% \noindent to include \emph{file} into your document. The optional parameter
% accepts a comma separated list of \emph{keys} and associated
% \emph{values}. The \emph{keys} can be used to alter the width, height
% and rotation of the included graphic. Table~\ref{keyvals} lists the
% most important keys.
\item \textsf{graphicx} 패키지를 전처리부에 로드하도록 지시한다.
\begin{lscommand}
  \verb|\usepackage{graphicx}|
\end{lscommand}
\item 문서의 본문에서 그림을 삽입할 위치에 다음 명령을 쓴다.
\begin{lscommand}
\ci{includegraphics}\verb|[|\emph{key}=\emph{value}, \ldots\verb|]{|\emph{file-name}\verb|}|
\end{lscommand}
\noindent 선택 인자는 \emph{key}와 이에 연결된 \emph{value}의 리스트를 쉼표로 구분하여 받아들인다. \emph{key}에 
width, height, rotation이 올 수 있다. 이 값들은 삽입할 그림의 너비, 높이, 회전을 나타낸다. 표~\ref{keyvals}에
중요한 key를 요약해놓았다.
% \end{enumerate}
\end{enumerate}

% \begin{table}[tb]
% \caption{Key Names for \textsf{graphicx} Package.}
% \label{keyvals}
% \begin{lined}{9cm}
% \begin{tabular}{@{}ll}
% \texttt{width}& scale graphic to the specified width\\
% \texttt{height}& scale graphic to the specified height\\
% \texttt{angle}& rotate graphic counterclockwise\\
% \texttt{scale}& scale graphic \\
% \end{tabular}

% \bigskip
% \end{lined}
% \end{table}

\begin{table}[!tb]
\caption{\textsf{graphicx} 패키지의 key}
\label{keyvals}
\begin{lined}{9.5cm}
\begin{tabular}{@{}ll}
\texttt{width}& 그림의 너비를 주어지는 값으로 맞춘다. \\
\texttt{height}& 그림의 높이를 주어지는 값으로 맞춘다. \\
\texttt{angle}& 그림을 반시계방향으로 주어진 각도만큼 회전시킨다.\\
\texttt{scale}& 그림을 비례적으로 늘리거나 줄인다.\\[-1ex]
\end{tabular}

\bigskip
\end{lined}
\end{table}

% The example code in figure~\ref{figureex} on page~\pageref{figureex} may help to clarify things.
%\pageref{figureex}페이지의 
그림~\ref{figureex}의 예시 코드를 보면 이해가 쉬울 것이다.
% \begin{figure}
% \begin{lined}{9cm}
% \begin{verbatim}
% \includegraphics[angle=90,width=\textwidth]{test.png}
% \end{verbatim}
% \end{lined}
% \caption{Example code for including \texttt{test.png} into a document.\label{figureex}}
% \end{figure}
\begin{figure}[!htb]
\begin{lined}{10cm}
\begin{verbatim}
\includegraphics[angle=90,width=.5\textwidth]{test.png}
\end{verbatim}
\end{lined}
\caption{\texttt{test.png}를 문서에 삽입하는 예시 코드\label{figureex}}
\end{figure}
% It includes the graphic stored in the file \texttt{test.png}. The
% graphic is \emph{first} rotated by an angle of 90 degrees and
% \emph{then} scaled to the final width of 0.5 times the width of a
% standard paragraph.  The aspect ratio is $1.0$, because no special
% height is specified.  The width and height parameters can also be
% specified in absolute dimensions. Refer to Table~\ref{units} on
% page~\pageref{units} for more information. If you want to know more
% about this topic, make sure to read \cite{graphics}.
이 코드는 \texttt{test.png}라는 이름의 그림을 포함하는 것이다.
그림을 \emph{먼저} 90도 회전시키고 \emph{그 다음에} 표준 문단 폭의 0.5배를 최종적인 너비로
설정하여 그림을 축소(확대)한다. 가로세로비는 $1.0$이다. 왜냐하면 높이(height) 값이 별도로 주어지지 않았기 때문이다.
width와 height 인자의 값으로는 절대 길이를 지정할 수도 있다. 길이에 대한 더 자세한 사항은 
\pageref{units}페이지의 표~\ref{units}\를 보라. 
그림 삽입에 관하여 더 알고 싶으면 \cite{graphics}\를 꼭 읽어보자.

% \section{Floating Bodies}
\section{떠다니는 개체} \label{floats}
% Today most publications contain a lot of figures and tables. These
% elements need special treatment, because they cannot be broken across
% pages.  One method would be to start a new page every time a figure or
% a table is too large to fit on the present page. This approach would
% leave pages partially empty, which looks very bad.
오늘날 대부분의 출판물에는 많은 그림과 표가 포함되어 있다. 이러한 요소들은 특별한 취급이 필요한데
그 이유는 이들을 잘라서 다른 페이지에 배치할 수 없기 때문이다.
한 가지 방법은 그림이나 표가 현재 페이지의 남은 공간에 다 들어가지 않을 때마다 새로운 페이지를 시작하는 것인데 
이렇게 하면 페이지에 빈 공간이 생기게 되고 결과적으로 보기 좋지 않다.

% The solution to this problem is to `float' any figure or table that
% does not fit on the current page to a later page, while filling the
% current page with body text. \LaTeX{} offers two environments for
% \wi{floating bodies}; one for tables and  one for figures.  To
% take full advantage of these two environments it is important to
% understand approximately how \LaTeX{} handles floats internally.
% Otherwise floats may become a major source of frustration, because
% \LaTeX{} never puts them where you want them to be.
이 문제에 대한 해결책은 현재 페이지 남은 부분에 배치할 수 없는 그림이나 표를 ``떠다니게''
하고 페이지의 남은 부분에 텍스트가 연속되어 흐르게 하는 것이다.
\LaTeX 은 이러한 \wi{떠다니는 개체}[floating bodies]를 위한 두 가지 환경을 마련하고 있다.
하나는 표를 위한 것이고 다른 하나는 그림을 위한 것이다.
이 두 가지 환경을 잘 활용하려면 \LaTeX 이 떠다니는 개체를 내부적으로 어떻게 다루는가를 
대강 이해하는 것이 중요하다.
그렇지 못하면 떠다니는 개체는 그저 골칫거리가 되고 말아서 \LaTeX 은 원하는 위치에 그림이 나오지 
않는다고 불평하게 하는 원인이 된다.

% \bigskip
% Let's first have a look at the commands \LaTeX{} supplies
% for floats:
\bigskip
먼저 떠다니는 개체를 위한 \LaTeX\ 명령을 살펴보자.

% Any material enclosed in a \ei{figure} or \ei{table} environment will
% be treated as floating matter. Both float environments support an optional
% parameter
\ei{figure}나 \ei{table} 환경으로 둘러싸인 부분은 떠다니는 개체로 간주된다.
% \begin{lscommand}
% \verb|\begin{figure}[|\emph{placement specifier}\verb|]| or
% \verb|\begin{table}[|\ldots\verb|]|
% \end{lscommand}
\begin{lscommand}
\verb|\begin{figure}[|\emph{placement specifier}\verb|]| 
  또는
\verb|\begin{table}[|\ldots\verb|]|
\end{lscommand}
% \noindent called the \emph{placement specifier}. This parameter
% is used to tell \LaTeX{} about the locations to which the float
% is allowed to be moved.  A \emph{placement specifier} is constructed by building a string
% of \emph{float-placing permissions}. See Table~\ref{tab:permiss}.
\noindent  두 환경에 선택 인자를 줄 수 있다. 이것을 \emph{위치지정자(place specifier)}라고 한다.
이 인자들은 개체를 이동시켜 놓을 위치를 \LaTeX 에게 알려주는 역할을 한다.
\emph{위치지정자}는 \emph{허용위치}를 나타내는 문자열로 이루어진다. 표~\ref{tab:permiss}\을 보라.%
\trfnote{%
  \ei{table}과 \ei{tabular}를 우리말로 둘 다 `표'라고 하다 보니 생기는 오해가 있다.
  \ei{tabular}는 행과 열에 맞추어서 요소를 배열하고 필요하다면 가로선과 세로선을 그은 식자 형태를 가리키는 것인 반면,
  \ei{table}은 그렇게 만들어진 \ei{tabular}를 둘러싸서 떠다니게 만드는 환경을 가리킨다.
  실제로 \ei{table} 환경 안에 오는 내용물은 반드시 표가 아니라도 상관없고 심지어 그림이 와도 된다.
  그러나 \cs{caption}을 붙이면 명칭이 ``표~1:'' 또는 ``Table~1:''과 같이 나타난다. \ei{figure}에 대해서도 같다.
}

% \begin{table}[!bp]
% \caption{Float Placing Permissions.}\label{tab:permiss}
% \noindent \begin{minipage}{\textwidth}
% \medskip
% \begin{center}
% \begin{tabular}{@{}cp{8cm}@{}}
% Spec&Permission to place the float \ldots\\
% \hline
% \rule{0pt}{1.05em}\texttt{h} & \emph{here} at the very place in the text
%   where it occurred.  This is useful mainly for small floats.\\[0.3ex]
% \texttt{t} & at the \emph{top} of a page\\[0.3ex]
% \texttt{b} & at the \emph{bottom} of a page\\[0.3ex]
% \texttt{p} & on a special \emph{page} containing only floats.\\[0.3ex]
% \texttt{!} & without considering most of the  internal parameters\footnote{Such as the
%     maximum number of floats allowed  on one page.}, which could otherwise stop this
%   float from being placed.
% \end{tabular}
% \end{center}
% \end{minipage}
% \end{table}
\begin{table}[!bp]
\caption{떠다니는 개체의 허용 위치}\label{tab:permiss}
\noindent \begin{minipage}{\textwidth}
\medskip
\begin{center}
\begin{tabular}{@{}cp{8cm}@{}}
지정자&개체가 놓이도록 허용되는 위치 \ldots\\
\hline
\rule{0pt}{1.05em}\texttt{h} & \emph{here}. 텍스트상의 현재 위치. 크지 않은 떠다니는 개체에 유용하다.\\[0.3ex]
\texttt{t} & \emph{top}. 페이지의 상단\\[0.3ex]
\texttt{b} & \emph{bottom}. 페이지의 하단\\[0.3ex]
\texttt{p} & \emph{page}. 그림과 표만으로 이루어지는 별도 페이지\\[0.3ex]
\texttt{!} & 내부 파라미터 값에 따르면 현재의 개체를 놓을 수 없을 때 내부 파라미터\footnote{예를 들면 한 페이지에 허용되는 떠다니는 개체의 수 상한값 등}를 무시하도록 함
\end{tabular}
\end{center}
\end{minipage}
\end{table}

% For example, a table could be started with the following line
% \begin{code}
% \verb|\begin{table}[!hbp]|
% \end{code}
% \noindent The \wi{placement specifier} \verb|[!hbp]| allows \LaTeX{} to
% place the table right here (\texttt{h}) or at the bottom (\texttt{b})
% of some page
% or on a special floats page (\texttt{p}), and all this even if it does not
% look that good (\texttt{!}). If no placement specifier is given, the standard
% classes assume \verb|[tbp]|.
예를 들면 다음 한 줄로 table이 시작된다.
\begin{code}
\verb|\begin{table}[!hbp]|
\end{code}
\noindent \wi{위치지정자} \verb|!hbp|는 이 표가 바로 이곳(\texttt{h})이나 페이지의 바닥(\texttt{b})
아니면 별도 페이지(\texttt{p})에 위치할 수 있도록 하면서 이 위치잡기가 그다지 만족스럽지 못한 경우라도(\texttt{!})
이행하라는 의미를 가지고 있다. 위치지정자가 주어지지 않았을 때 표준 클래스에서는 \verb|[tbp]|를 쓰도록 정해져 있다.

% \LaTeX{} will place every float it encounters according to the
% placement specifier supplied by the author. If a float cannot be
% placed on the current page it is deferred either to the
% \emph{figures} queue or the \emph{tables} queue.\footnote{These are FIFO---`first in first out'---queues!}  When a new page is started,
% \LaTeX{} first checks if it is possible to fill a special `float'
% page with floats from the queues. If this is not possible, the first
% float on each queue is treated as if it had just occurred in the
% text: \LaTeX{} tries again to place it according to its
% respective placement specifiers (except `h,' which is no longer
% possible).  Any new floats occurring in the text get placed into the
% appropriate queues. \LaTeX{} strictly maintains the original order of
% appearance for each type of float. That's why a figure that cannot
% be placed pushes all further figures to the end of the document.
% Therefore:
\LaTeX 은 떠다니는 개체를 만날 때마다 문서작성자가 제공한 위치지정자에 맞추어서 위치를 정한다.
한 개체가 현재 페이지에서 자리를 잡지 못하면 그것은 \emph{figures}나 \emph{tables} 대기열에
넣어서 처리를 미룬다.\footnote{FIFO, 즉 ``먼저 들어온 것이 먼저 나가는'' 큐(queue)이다.}
\LaTeX 은 대기열에 있는 개체들이 특별한 `플로트가 놓이는' 페이지를 만들기에 충분한지를 검사한다.
이것이 가능하다면 대기열의 첫 번째 개체를 바로 그 위치에 개체가 불린 것처럼 취급한다. 즉
\LaTeX 은 다시 그 개체에 주어져 있는 위치지정자의 요구에 따라서 위치를 결정하려 한다. (단 `h'
지정자는 더이상 불가능하다.)
텍스트에서 새로운 떠다니는 개체가 발생할 때마다 적절한 대기열 속에 들어간다.
\LaTeX 은 개체의 유형별로 출현 순서를 엄격히 유지한다. 그러다 보면 문서의 끝까지
더이상 그림의 자리를 잡지 못하고 미루어지는 경우가 생기는 것이다.
그러므로,

% \begin{quote}
% If \LaTeX{} is not placing the floats as you expected,
% it is often only one float jamming one of the two float queues.
% \end{quote}
\begin{quote}
  만약 \LaTeX 이 그림이나 표를 원하는 곳에 출력해주지 않는다면
  두 플로트 대기열 가운데 하나에서 어떤 개체가 처리되지 않은 채로 정체되어 있기 때문이다.
\end{quote}


% While it is possible to give \LaTeX{}  single-location placement
% specifiers, this causes problems.  If the float does not fit in the
% location specified it becomes stuck, blocking subsequent floats.
% In particular, you should never, ever use the [h] option---it is so bad
% that in more recent versions of \LaTeX, it is automatically replaced by
% [ht].
\LaTeX 에게 위치지정자를 하나만 전달하는 것도 가능하지만 이렇게 하면 문제가 생긴다.
만약 플로트가 그 지정된 위치에 놓이는 데 실패하면 바로 정체되어 그 이후에 오는 플로트들이
처리되지 못하게 만든다.
특히, [h] 옵션 하나만 지정되어 있다면 아주 곤란하기 때문에 요즘 \LaTeX 은 문서작성자가 
[h]라고만 하더라도 자동으로 [ht]로 바꾸어서 처리한다.

% \bigskip
% \noindent Having explained the difficult bit, there are some more things to
% mention about the \ei{table} and \ei{figure} environments.
% Use the
\bigskip
\noindent 좀 어려운 부분을 설명하였고, \ei{table}과 \ei{figure} 환경에 대해 
언급할 것이 몇 가지 더 있다.

% \begin{lscommand}
% \ci{caption}\verb|{|\emph{caption text}\verb|}|
% \end{lscommand}
\begin{lscommand}
\ci{caption}\verb|{|\emph{caption text}\verb|}|
\end{lscommand}
% \noindent command to define a caption for the float. A running number and
% the string ``Figure'' or ``Table'' will be added by \LaTeX.
\noindent 이 명령은 떠다니는 개체에 캡션을 달아준다. ``Figure'' 또는 ``Table''이라는 
문자열\trfnote{%
  한국어 텍에서는 각각 ``그림''과 ``표''로 식자된다.
}%
이 붙은 그림번호 또는 표번호가 \LaTeX 에 의해 부여된다.

\bigskip

다음 두 명령은 그림 목차와 표 목차를 생성한다.
% The two commands

% \begin{lscommand}
% \ci{listoffigures} and \ci{listoftables}
% \end{lscommand}
\begin{lscommand}
\ci{listoffigures} \ci{listoftables}
\end{lscommand}

% \noindent operate analogously to the \verb|\tableofcontents| command,
% printing a list of figures or tables, respectively.  These lists will
% display the whole caption, so if you tend to use long captions
% you must have a shorter version of the caption for the lists.
% This is accomplished by entering the short version in brackets after
% the \verb|\caption| command.
% \begin{code}
% \verb|\caption[Short]{LLLLLoooooonnnnnggggg}|
% \end{code}
\noindent 이 명령은 \verb|\tableofcontents| 명령과 같은 방식으로 동작하여
각각 그림과 표의 목차를 인쇄한다.
이 목차에는 캡션이 식자되는데 만약 캡션이 길어서 문제가 되면 목차에 나타날 
짧은 캡션을 지정해두어야 한다.
\verb|\caption| 명령의 대괄호 선택 인자에 짧은 캡션을 적어두는 방식으로 하면 된다.
\begin{code}
\verb|\caption[Short]{LLLLLoooooonnnnnggggg}|
\end{code}

% Use \ci{label} and \ci{ref} to create a reference to a float within
% your text. Note that the \ci{label} command must come \emph{after} the
% \ci{caption} command since you want it to reference the number of the
% caption.
\ci{label}과 \ci{ref}는 문서 안에서 특정 플로트에 대하여 참조할 수 있게 해준다.
\ci{label} 명령은 반드시 \ci{caption} 명령 \emph{이후에} 써야 한다는 것을 기억하라.
왜냐하면 참조할 숫자(번호)가 캡션이 생성될 때 만들어지기 때문이다.

% The following example draws a square and inserts it into the
% document. You could use this if you wanted to reserve space for images
% you are going to paste into the finished document.
다음 보기는 사각형을 그려서 문서에 삽입하고 있다.
문서에 그림 넣을 공간을 만든 다음에 최종적으로 출력하여 그림을 풀로 붙여넣을 
생각이라면 이 코드를 이용할 수 있다.

% \begin{code}
% \begin{verbatim}
% Figure~\ref{white} is an example of Pop-Art.
% \begin{figure}[!hbtp]
% \includegraphics[angle=90,width=\textwidth]{white-box.pdf}
% \caption{White Box by Peter Markus Paulian.\label{white}}
% \end{figure}
% \end{verbatim}
% \end{code}
\begin{code}
\begin{verbatim}
Figure~\ref{white} is an example of Pop-Art.
\begin{figure}[!hbtp]
\includegraphics[angle=90,width=\textwidth]{white-box.pdf}
\caption{White Box by Peter Markus Paulian.\label{white}}
\end{figure}
\end{verbatim}
\end{code}

% \noindent In the example above,
% \LaTeX{} will try \emph{really hard}~(\texttt{!})\ to place the figure
% right \emph{here}~(\texttt{h}).\footnote{assuming the figure queue is
%   empty.} If this is not possible, it tries to place the figure at the
% \emph{bottom}~(\texttt{b}) of the page.  Failing to place the figure
% on the current page, it determines whether it is possible to create a float
% page containing this figure and maybe some tables from the tables
% queue. If there is not enough material for a special float page,
% \LaTeX{} starts a new page, and once more treats the figure as if it
% had just occurred in the text.
\noindent 이 예에서 
\LaTeX 은 바로 이곳 (\texttt{h})에 그림을 놓으려고 \emph{할 수 있는 한} (\texttt{!})
시도해 본다.\footnote{대기열에 다른 그림이 없다고 가정.}
만약 이것이 불가능하면 그림을 페이지의 \emph{바닥} (\texttt{b})에 놓으려 한다.
현재 페이지에서 그것이 불가능하면 (표 대기열에 혹시 표가 있다면 그것과 함께) 이 그림을 
놓을 별도의 플로트 페이지를 만들 수 있을지를 결정한다. 
개체들이 페이지를 채울 수 있을 정도가 되지 않는다면 \LaTeX 은 새 페이지를 시작하여
이 위치에서 그림이 주어진 것처럼 한 번 더 처리 과정을 시도한다.

% Under certain circumstances it might be necessary to use the
어떤 경우에는 다음 명령을 써야할 때가 있다.

% \begin{lscommand}
% \ci{clearpage} or even the \ci{cleardoublepage}
% \end{lscommand}
\begin{lscommand}
\ci{clearpage} 또는 \ci{cleardoublepage}
\end{lscommand}

% \noindent command. It orders \LaTeX{} to immediately place all
% floats remaining in the queues and then start a new
% page. \ci{cleardoublepage} even goes to a new right-hand page.
\noindent \LaTeX 에게 즉시 대기열에 남아 있는 모든 떠다니는 개체를 
식자하여 대기열을 비우고 새로운 페이지를 열라는 것이다. \ci{cleardoublepage}는 
새로 열리는 페이지가 홀수페이지(오른쪽 면)가 되도록 필요하다면 짝수면을 추가하여 새로운 페이지로 
이동하라는 의미이다.\trfnote{%
  \ci{newpage}는 떠다니는 개체를 즉시 청산(flush)하도록 요구하지 않는다.
}

% % Local Variables:
% % TeX-master: "lshort2e"
% % mode: latex
% % mode: flyspell
% % End:
